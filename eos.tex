\documentclass[aps,prc,reprint,amsmath,nofootinbib]{revtex4-1}

\usepackage{hyperref}
\usepackage{graphicx}
\graphicspath{{fig/}}

\usepackage{mdwlist}

\newcommand{\nch}{N_\text{ch}}

\begin{document}

\title{Hydrodynamic simulations of heavy-ion collisions with different equations of state}

\author{J.\ Scott Moreland}
\affiliation{Department of Physics, Duke University, Durham, NC 27708-0305}
\author{Ron A.\ Soltz}
\affiliation{Lawrence Livermore National Laboratory, Livermore, CA 94551-0808}

\date{\today}

\begin{abstract} 
   The QCD Equation of State (EoS) is an essential ingredient for the hydrodynamic models used to study heavy ion collisions.  
   Recent results by the HotQCD and Wuppertal-Budapest collaborations lattice gauge calculations of the QCD EoS at the continuum 
   limit show good agreement within errors.  However it is unknown whether current errors are sufficient for current simulations 
   or whether further improvements are needed.  We explore this question by performing hydrodynamic calculations with the VISHNU 2+1D 
   hydrodynamic code with fluctuating initial conditions and UrQMD cascade code for the two EoS calculations and a sampling of EoS curves 
   within the given errors.  Comparisons are made to spectra ($\pi$, K, p), flow ($v_2, v_3$), and azimuthally averaged HBT radii for 
   200~GeV Au+Au collisions.
\end{abstract}

\maketitle

\section{Introduction}
At temperatures exceeding two trillion Kelvin, quantum chromodynamics predicts a nuclear phase transition  from bound states of colorless hadrons
to a deconfined soup of quarks and gluons known as a quark-gluon plasma (QGP). QGP is produced and studied in the laboratory by colliding nuclei at 
ultrarelativistic energies to excite matter beyond the predicted phase transition temperature and produce a droplet of QGP which quickly cools and freezes into 
hadrons.



on the order of $\sim 10^{-23}$ s, cannot be observed directly and is typically
studied using computer simulations to reconstruct the full spacetime evolution of the collision.

Viscous relativistic fluid dynamics is a popular theoretical framework to describe the transport dynamics of the liquid-like QGP produced in each collision,



and match the hot and dense early phases of the collision to later times where it can be matched to experimental observables.




\section{Introduction}


\section{Model}

\section{Acknowledgements}

\medskip
JSM acknowledges support by the DOE/NNSA Stockpile Stewardship Graduate Fellowship under grant no.~DE-FC52-08NA28752.

%\bibliography{trento,duke-qcd-refs/Duke_QCD_refs}


\end{document}
