\documentclass[aps,prc,reprint,amsmath,nofootinbib,superscriptaddress]{revtex4-1}

\usepackage{hyperref}
\usepackage{graphicx}
\usepackage{amsmath}
\usepackage{cleveref}
\graphicspath{{fig/}}

\usepackage{mdwlist}

\newcommand{\nch}{N_\text{ch}}

\begin{document}

\title{Hydrodynamic simulations of relativistic heavy-ion collisions\\ with different calculations of the QCD equation of state}

\author{J.\ Scott Moreland}
\affiliation{Department of Physics, Duke University, Durham, NC 27708-0305}
\author{Ron A.\ Soltz}
\affiliation{Lawrence Livermore National Laboratory, Livermore, CA 94551-0808}

\date{\today}

\begin{abstract} 
   
\end{abstract}

\maketitle

\section{Introduction}

Fluid dynamics is a useful framework to study the collective behaviour of hot and dense nuclear matter produced in relativistic heavy-ions. 
Quantum Chromodynamics (QCD) predicts that at sufficiently high energies these collisions form a new state of matter consisting of deconfined 
quarks and gluons known as a quark-gluon plasma (QGP). Simulations based on relativistic viscous hydrodynamics play a central role in extracting 
properties of the QGP which expands and freezes into hadrons too quickly for direct observation.

The hydroynamic transport equations require two essential ingredients to specify the full time evolution of the QGP fireball: initial conditions 
which describe the thermal profile of the QGP droplet at some early starting time and a QCD equation of state which interrelates energy density, 
pressure and temperature of each fluid cell in local thermal equillibrium.

Lattice discretization is the only reliable method to calculate the QCD equation of state in the vicinity of the QGP phase transition and hence 
constitutes a critical component of hydrodynamic simulations. While lattice techniques are rigorous in their treatment of the underlying QCD 
Lagrangian, they are subject to numerical errors inherent in the lattice discretization procedure. These errors are manifest in differences in the 
continuum extrapolated QCD trace anomaly predicted by different lattice collaborations and lead to an overall uncertainty in the true value of the 
QCD equation of state at zero baryochemical potential.

Simulations using a lattice based equation of state inherrit all forms of numeric and systematic uncertainty associated with the underlying 
lattice methodology. These modeling uncertainties have been studied both at low temperature, by comparing simulations with a lattice equation of 
state to results obtained from a hadron resonance gas model \cite{Huovinen:2009yb}, and at high temperature by comparing the effect of different 
lattice parameterizations of the lattice equation of state on particle spectra and flow \cite{Huovinen:2005gy, Huovinen:2009yb}. 

%A notable exception to this convention is a recent model-to-data analysis 
%which parameterized the QGP equation of state and used a Bayesian, data driven approach to constrain its functional form \cite{Novak:2013bqa}. 

Recent calculations by the HotQCD and Wuppertal-Budapest collaborations of the QCD trace anomaly in the continuum limit now show good agreement within 
errors. This signals an important convergence in lattice descriptions of the QCD equation of state which previously exhibited a tension in the peak 
of the trace anomaly near the QGP phase transition. It is not yet clear however, if current lattice errors are under sufficient control for 
hydrodynamic transport models or if further improvement is needed.  

In this work, we analyze the effect of lattice errors on simulations of relativistic heavy-ion collisions by comparing simulation predictions obtained 
with different calculations of the QCD equation of state. We study the latest HotQCD and Wuppertal-Budapest equations of state, an older 
S95p-v1 parametrization constructed from HotQCD measurements with a coarser lattice spacing, and also a sampling of curves drawn from HotQCD published errors. The 
flows, spectra and Bertsch-Pratt radii are calculated for each equation of state and used to assess the need for further refinements to the QCD 
equation of state at zero baryo-chemical potential.

\section{Equations of State}

The hybrid approach adopted in this work uses the UrQMD transport model to simulate microscopic interactions once the system expands and freezes into hadrons extend the 
simulation to dillute regions of the system's spacetime evolution which are beyond the applicability of hydrodynamics. 


While the QCD equation of state enters the hydrodynamic phase of the simulation as a freely specified function interrelating energy density, pressure and temperature, its description in the 
kinetic phase of the collision is fixed -- determined by the finite number of particles and particle resonances included in the microscopic transport model.

As a result, we limit our study to differences in the QCD equation of state \emph{above} the QGP transition temperature where hydrodynamics allows us to 
freely vary its chosen form. We study three different parameterizations for this high temperature dependence -- two state of the art calculations 
in 2+1 flavour QCD from the HotQCD \cite{Bazavov:2014pvz} and Wuppertal-Budapest \cite{Borsanyi:2013bia} collaborations, as well as the older s95p-v1 
parameterization \cite{Huovinen:2009yb} constructed using lattice data measured with a coarser lattice spacing \cite{Bazavov:2009zn}.

The QCD equation of state is frequently characterized by the trace of the energy-momentum tensor, also referred to as the trace anomaly or interaction measure. 
When scaled by powers of the temperature, the trace anomaly forms a dimensionless measure
\begin{equation}
 I \equiv \frac{\Theta^{\mu\mu}(T)}{T^4} = \frac{e - 3p}{T^4},
\end{equation}
where $e$ is the local fluid energy density, $p$ the pressure and $T$ the temperature.

In Fig.~\ref{fig:trace} we plot the temperature scaled interaction measure of each equation of state as well as that of a hadron resonance gas calculated from 
the list of partial resonances included in the UrQMD collision kernel. The s95p-v1 parameterization, which was constructed to interpolate between 
a hadron resonance gas calculation at low temperatures and lattice results at high temperatures, is in good agreement with the UrQMD equation of state while the 
HotQCD and Wuppertal-Budapest results are slightly higher in the vicinity of the phase transition.

To ensure a self consistent description in regions of the collision where the simulation switches from hydrodynamics to Boltzmann transport, we match each high 
temperature lattice equation of state with the low temperature UrQMD equation of state. We thus define a piecewise function for the temperature scaled interaction measure,
\begin{equation}
 \label{interaction}
 I(T) =
  \begin{cases}
   I_\text{hrg}(T)	& T \le T_1 \\
   I_\text{blend}(T)	& T_1 < T < T_2 \\ 
   I_\text{lattice}(T)	& T \ge T_2,
  \end{cases}
\end{equation}
where $I_\text{hrg}$ is the hadron resonance gas trace anomaly in UrQMD pictured in Fig.~\ref{fig:trace}, $I_\text{lattice}$ represents one of the HotQCD, Wuppertal-Budapest or S95p-v1 
parameterizations and $I_\text{blend}$ is a function which smoothly connects between the two in the temperature interval $T_1 < T < T_2$,
\begin{equation}
  \label{interpolation}
  I_\text{blend} = (1-z)\, I_\text{hrg} + z\, I_\text{lattice}.
\end{equation}
The interpolation parameter $z \in [0,1]$ in equation \ref{interpolation} is constructed to match the first and second derviatives at the endpoints of the 
interpolation interval,
\begin{eqnarray}
 \label{smoothstep}
 \cr z &=& 6 x^5 - 15 x^4 + 10 x^3 \\
  \text{where } x &=& (T - T_1)/(T_2 - T_1).
\end{eqnarray}

\begin{figure}[t]
  \includegraphics[width=\columnwidth]{./fig/trace}
  \caption{\label{fig:trace} The temperature scaled QCD trace anomaly for the UrQMD. HotQCD, WB and s95p-v1 parameterizations as a function of temperature \cite{?}.}
\end{figure}

\begin{figure}[b]
  \includegraphics[width=\columnwidth]{./fig/trace_final}
  \caption{\label{fig:trace_final} The modified QCD trace anomalies HotQCD', WB' and S95' obtained from equation \eqref{interaction} and the corresponding lattice
	  parameterizations in Fig.~\ref{fig:trace}. The gray, vertical line marks the hydro-to-micro switching temperature $T_\text{sw} = 154$ MeV.}
\end{figure}

We fix the piecewise temperature bounds $T_1 = 154$ MeV and $T_2 = 180$ MeV for the hydro-to-micro switching temperature $T_\text{sw} = 154$ MeV used in this work.
While lower switching temperatures permit a smoother transition between the lattice and UrQMD equations of state, they necessitate chemical non-equillibirum corrections 
which occur naturally in UrQMD but are not accounted for in the present lattice parameterizations \cite{?}. The modified interaction measures, labeled with a prime to 
distinguish them from the raw lattice results, are plotted in Fig.~\ref{fig:trace_final}. The vertical gray line marks the hydro-to-micro switching temperature 
$T_\text{sw}=154$ MeV where the model switches from the VISH2+1 hydrodynamics code to UrQMD. 

In Fig.~\ref{fig:cs} we plot the squared speed of sound $c_s^2 = dp / de$ for each modified interaction measure. The speed of sound of the HotQCD' and WB' equations
of state are in good agreement while the S95' parameterization remains softer in a wider interval about the QGP phase transition. We note that the speed of sound in 
the HotQCD' and WB' parameterizations is clearly affected by the parametric transition \eqref{interpolation} in the vicinity of the switching temperature (vertical gray line), 
but that the imposed matching maintains continuity.

With the trace anomalies in hand, the energy density, pressure and entropy density are easily interrelated to specify the equation of state used in the analysis,
\begin{eqnarray}
 \cr \frac{p(T)}{T^4} &=& \int\limits_0^T dT'\, \frac{I(T')}{T'}, \\
 \frac{e(T)}{T^4} &=& I(T) + 3\, \frac{p(T)}{T^4}, \\
 \frac{s(T)}{T^3} &=& \frac{e(T) + p(T)}{T^4}. 
\end{eqnarray}
 

\section{Hybrid Model}

The comprehensive hybrid model used in this analysis is explained in detail in reference  \cite{} to simulate the spacetime evolution of each QGP droplet and 

We initial conditions are furnished using are 
furnished using a simplistic Monte Carlo Glauber model which incorporates event-by-event fluctuations in the positions of nucleons within the colliding nuclei.

\subsection{Initial Conditions}

The Monte Carlo Glauber model generate hydrodyamic initial conditions using a standard two-component ansatz which deposits entropy proportional to a linear combination 
of nucleon participants and binary nucleon-nucleon collisions,
\begin{equation}
 dS/dy \,\vert_{y=0} \propto \frac{(1-\alpha)}{2}N_\text{part} + \alpha N_\text{coll}
 \label{twocomponent}
\end{equation}
where for the binary collision fraction, we use $\alpha=0.14$ which has been shown to provide a good description of the centrality dependence of charged particle 
multiplicity in $200$ GeV gold-gold collisions \cite{?}.

The entropy is localized about each nucleon's transverse parton density $T_p({\bf x})$,
\begin{eqnarray}
 dS/dy \,\vert_{y=0} &\propto& \sum\limits_{i=0}^{N_\text{part,A}} w_i\, T_p({\bf x} - {\bf x}_i)(1-\alpha + \alpha\, N_\text{coll,i}) \nonumber \\
                     &+& \sum\limits_{j=0}^{N_\text{part,B}} w_j\, T_p({\bf x} - {\bf x}_i)(1-\alpha + \alpha\, N_\text{coll,j}),
 \label{glauber}
\end{eqnarray}
where the summations run over the participants in each nucleus, $N_\text{coll,i}$ denotes the number of binary collisions suffered by the $i^\text{th}$ nucleon 
and the proton density $T_p({\bf x})$ is described by a Gaussian
\begin{equation}
 T_p({\bf x}) = \frac{1}{\sqrt{2 \pi B}} \exp \left(-\frac{x^2+y^2}{2 B} \right)
\end{equation}
with transverse area $B = 0.36$ $\text{fm}^2$.

The random nucleon weights $w_i$ in equation \eqref{glauber} are sampled independently from a Gamma distribution with unit mean
\begin{equation}
 P_k(w) = \frac{k^k}{\Gamma(k)} w^{k-1} e^{-k w},
\end{equation}
and shape parameter $k = \text{Var}(P)^{-1}$ which modulates the variance of the distribution. 
These fluctuations are typically added \cite{?} to reproduce the large multiplicity fluctuations observed in minimum bias proton-proton collisions. 
In this work the shape parameter is fixed to $k=1$ determined by a fit to the $200$ GeV UA5 data \cite{?}. 

The initial condition profiles, which provide the entropy density $dS/(d^2r_\perp\, d\eta\, \tau_\text{therm})$ at the QGP thermalization time, are finally 
rescaled by an overall normalization factor to fit the measured charged particle multiplicity in $0$--$10\%$ centrality collisions.

\begin{figure}
  \includegraphics[width=\columnwidth]{./fig/cs}
  \caption{\label{fig:cs} Speed of sound squared $c_s^2$ plotted versus temperature $T$ for the three equations of state used in this study. The vertical
	   gray line indicates the switching temperature $T_\text{sw} = 154$ MeV where the model switches from fluid dynamics to a microscopic transport model.}
\end{figure}

\subsection{Hydrodyamics and Boltzmann Transport}

We follow the work in reference \cite{?} and fix the hydrodynamic specific shear viscosity to a constant value $(\eta/s)_\text{QGP}=0.08$ which has been shown to 
provide a reasonable description of measured spectra and flows for the system at hand. For the purposes of this study, we assume vanishing bulk viscosity 
$\zeta/s=0$, although it would be interesting to account for finite bulk viscosity in future work, as its functional form is sensitive to the peak of the QCD 
trace anomaly near the QGP phase transition \cite{?}.

In order to switch from hydrodyamic field equations to microscopic transport, the VISHNU model asserts a sudden switching temperature $T_\text{sw}$ at which the 
hydrodynamic energy-momentum tensor is particlized using the Cooper-Frye freezeout prescription,
\begin{equation}
 E\frac{dN_i}{d^3p} = \int_\sigma f_i(x,p) p^\mu d^3\sigma_\mu
 \label{cooper-frye}
\end{equation}
where $f_i$ is the distribution function of particle species $i$, $p^\mu$ is its four-momentum and $d^3\sigma_\mu$ characterizes an element of the isothermal freezeout 
hypersurface defined by $T_\text{sw}$.

We choose a switching temperature $T_\text{sw} = 154$ MeV which coincides with the temperature where the lattice equations of state is constructed to match 
the UrQMD equation of state.



\section{Postprocessing and Analysis}

\section{Results}

\begin{figure*}[t]
  \includegraphics[width=\textwidth]{./fig/spectra}
  \caption{
    \label{fig:spectra}
  }
\end{figure*}

\begin{figure*}[t]
  \includegraphics[width=\textwidth]{./fig/v2}
  \caption{
    \label{fig:spectra}
  }
\end{figure*}

\begin{figure*}[t]
  \includegraphics[width=\textwidth]{./fig/v3}
  \caption{
    \label{fig:spectra}
  }
\end{figure*}

\begin{figure*}[t]
  \includegraphics[width=\textwidth]{./fig/hbt}
  \caption{
    \label{fig:spectra}
  }
\end{figure*}

\section{Conclusion}

\section{Summary}

\begin{acknowledgments}
 JSM acknowledges support by the DOE/NNSA Stockpile Stewardship Graduate Fellowship under grant no.~DE-FC52-08NA28752.
\end{acknowledgments}

\bibliography{eos,duke-qcd-refs/Duke_QCD_refs}


\end{document}
