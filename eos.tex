\documentclass[aps,prc,reprint,amsmath,nofootinbib,superscriptaddress]{revtex4-1}

\usepackage{hyperref}
\usepackage{graphicx}
\graphicspath{{fig/}}

\usepackage{mdwlist}

\newcommand{\nch}{N_\text{ch}}

\begin{document}

\title{Hydrodynamic simulations of heavy-ion collisions with different equations of state}

\author{J.\ Scott Moreland}
\affiliation{Department of Physics, Duke University, Durham, NC 27708-0305}
\author{Ron A.\ Soltz}
\affiliation{Lawrence Livermore National Laboratory, Livermore, CA 94551-0808}

\date{\today}

\begin{abstract} 
   
\end{abstract}

\maketitle

\section{Introduction}

Hydrodynamic simulations are a popular tool to model the spacetime evolution of the quark-gluon plasma (QGP) produced in relativistic heavy-ion collisions.
Lattice regularization is the only reliable method to calculate the QCD equation of state equation of state in the vicinity of a phase transition and hence 
constitutes a critical component of modern computer simulations. While lattice techniques are rigorous in their treatment of the underlying QCD Lagrangian, 
they are subject to numerical errors inherent in the lattice discretization procedure. These errors are manifest in differences in the continuum extrapolated QCD 
trace anomaly predicted by different lattice collaborations and lead to an overall uncertainty in the true value of the QCD equation of state at zero baryochemical 
potential.

For the purposes of hydrodynamic simulations, the equation of state is typically treated as a theoretically constrained quantity in contrast to e.g. the
QGP specific shear viscosity $\eta/s$ which is varied and tuned to optimally replicate experimental data. Consequently, numerical discrepancies between different 
lattice collaborations introduce an inherent systematic bias in the best fit values of underconstrained QGP properties determined from systematic model-to-data comparison. 
A notable exception to this convention is a recent model-to-data analysis which parameterized the QGP equation of state and used a Bayesian, data driven approach 
to constrain its functional form \cite{Novak:2013bqa}. 

Uncertainties in the equation of state have been studied both at low temperature, by comparing lattice predictions to results from a hadron resonance gas model
\cite{Huovinen:2009yb}, and at high temperature by comparing hydrodynamic predictions obtained using different parameterizations of the QCD trace anomaly 
\cite{Huovinen:2005gy, Huovinen:2009yb}. Large differences in the flows and spectra are observed when switching from a bag model equation of state with a first-order 
phase transition to a lattice equation of state exhibiting a smooth crossover, but only mild sensitivity is observed to different parametric forms of the lattice 
calculations \cite{Huovinen:2009yb}.

In this work we study the lattest HotQCD and Wuppertal-Budapest 


\bibliography{eos,duke-qcd-refs/Duke_QCD_refs}


\section{Lattice Uncertainties}

\section{Equations of state}


\section{Hydrodynamic Model}

\section{Acknowledgements}

\medskip
JSM acknowledges support by the DOE/NNSA Stockpile Stewardship Graduate Fellowship under grant no.~DE-FC52-08NA28752.

%\bibliography{trento,duke-qcd-refs/Duke_QCD_refs}


\end{document}
