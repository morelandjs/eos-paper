\documentclass[aps,prc,reprint,amsmath,nofootinbib,superscriptaddress]{revtex4-1}

\usepackage{hyperref}
\usepackage{graphicx}
\graphicspath{{fig/}}

\usepackage{mdwlist}

\newcommand{\nch}{N_\text{ch}}

\begin{document}

\title{Quantifying uncertainty in hydrodyamic simulations of heavy-ion collisions attributable to imperfect knowledge of the QCD equation of state}

\author{J.\ Scott Moreland}
\affiliation{Department of Physics, Duke University, Durham, NC 27708-0305}
\author{Ron A.\ Soltz}
\affiliation{Lawrence Livermore National Laboratory, Livermore, CA 94551-0808}

\date{\today}

\begin{abstract} 
   
\end{abstract}

\maketitle

\section{Introduction}

Hydrodynamic simulations are a popular tool to model the spacetime evolution of the quark-gluon plasma (QGP) produced in relativistic heavy-ion collisions.
Lattice regularization is the only reliable method to calculate the QCD equation of state equation of state in the vicinity of a phase transition and hence 
constitutes a critical component of modern computer simulations. While lattice techniques are rigorous in their treatment of the underlying QCD Lagrangian, 
they are subject to numerical errors inherent in the lattice discretization procedure. These errors are manifest in differences in the continuum extrapolated QCD 
trace anomaly predicted by different lattice collaborations and lead to an overall uncertainty in the true value of the QCD equation of state at zero baryochemical 
potential.

For the purposes of hydrodynamic simulations, the equation of state is typically treated as a theoretically constrained quantity in contrast to e.g. the
QGP specific shear viscosity $\eta/s$ which is varied and tuned to optimally replicate experimental data. Consequently, numerical discrepancies between different 
lattice collaborations introduce an inherent systematic bias in the best fit values of underconstrained QGP properties determined from systematic model-to-data comparison. 
A notable exception to this convention is a recent model-to-data analysis which parameterized the QGP equation of state and used a Bayesian, data driven approach 
to constrain its functional form \cite{Novak:2013bqa}. 

Uncertainties in the equation of state have been studied both at low temperature, by comparing lattice predictions to results from a hadron resonance gas model
\cite{Huovinen:2009yb}, and at high temperature by comparing hydrodynamic predictions obtained using different parameterizations of the QCD trace anomaly 
\cite{Huovinen:2005gy, Huovinen:2009yb}. Large differences in the flows and spectra were observed when switching from a bag model equation of state with a first-order 
phase transition to a lattice equation of state exhibiting a smooth crossover, but only mild variations were observed when switching between different lattice 
parameterizations with a smooth crossover \cite{Huovinen:2009yb}.

Recent calculations by the HotQCD and Wuppertal-Budapest collaborations of the QCD trace anomaly in the continuum limit now show good agreement within errors. This signals an important
convergence in lattice descriptions of the QCD equation of state which previously exhibited a tension in the peak of the trace anomaly near the QGP phase transition. It is not yet clear however, if current
lattice errors are under sufficient control for hydrodynamic transport models or if further improvement is needed.  

In this work, we analyze the current status of lattice gauge calculations in the continuum limit by comparing simulation predictions using different lattice calculations for the QGP equation of state.
We study the latest HotQCD and Wuppertal-Budapest parametrizations as well as the depreciated s95 parametrization based on older HotQCD lattice results. 

We embed each equation of state in an event-by-event hydrodynamic model with a hadronic afterburner and measure spectra, flows and Bertsch-Pratt radii predicted by the simulations to quantify systematic 
differences between the calculations. We also assess the generic uncertainty introduced by continuum extrapolation when constructing the best fit parametrization by sampling splines from the bootstrap coefficients 
used in the HotQCD error analysis. Using these results, we comment on the resolving power of hydrodynamic simulations and assess the need for improved lattice calculations at zero baryo-chemical potential.

\bibliography{eos,duke-qcd-refs/Duke_QCD_refs}


\section{Lattice Uncertainties (Ron)}

\section{Methodology}

We use the {\ttfamily VISHNU} event-by-event hybrid model which couples the boost invariant {\ttfamily VISH2+1} viscous hydrodynamics code to the {\ttfamily UrQMD} hadronic transport model. The initial conditions are obtained
from a standard two-component Glauber Monte-Carlo which deposits entropy proportional to a linear combination of wounded nucleons and binary collisions,
\begin{equation}
 dS/dy \,\vert_{y=0} \propto \frac{(1-\alpha)}{2}N_\text{part} + \alpha N_\text{coll}.
\end{equation}

We choose a simplistic implementation of the Monte Carlo Glauber model and deposit wounded nucleon and binary collision entropy proportional to the transverse proton density $T_p({\bf x})$,
\begin{eqnarray}
 dS/dy \,\vert_{y=0} &\propto& \sum\limits_{i=0}^{N_\text{part,A}} w_i\, T_p({\bf x} - {\bf x}_i)(1+\alpha\, N_\text{coll,i}) \nonumber \\
                     &+& \sum\limits_{j=0}^{N_\text{part,B}} w_j\, T_p({\bf x} - {\bf x}_i)(1+\alpha\, N_\text{coll,j}).
\end{eqnarray}
where $N_\text{coll,i}$ denotes the number of binary collisions suffered by the $i^\text{th}$ nucleon and $w_i$ is a random weight sampled from a Gamma distribution with unit mean.
\begin{equation}
 P_k(w) = \frac{k^k}{\Gamma(k)} w^{k-1} e^{-k w},
\end{equation}
The shape parameter $k=1$ is tuned to fit the slope of the minimum bias p+p multiplicity distribution at $200$ GeV and introduces additional fluctuations on top of the nucleon position fluctuations.

For the wounded nucleon/binary collision mixture we choose the rather common value $\alpha=0.14$ which yields a good description of the centrality dependent charged particle multiplicity 
in $200$ GeV Au+Au collisions after full hydrodynamic and hadronic evolution. The transverse proton density $T_p({\bf x})$ is fixed using a Gaussian of transverse area $B = 0.36$ $\text{fm}^2$,
\begin{equation}
 T_p({\bf x}) = \frac{1}{\sqrt{2 \pi B}} \exp \left(-\frac{x^2+y^2}{2 B} \right).
\end{equation}
 


\section{Results}

\begin{figure}[t]
  \includegraphics[width=\columnwidth]{./fig/eos_comparison}
  \caption{
    \label{fig:spectra}
  }
\end{figure}

\begin{figure*}[t]
  \includegraphics[width=\textwidth]{./fig/spectra}
  \caption{
    \label{fig:spectra}
  }
\end{figure*}

\begin{figure*}[t]
  \includegraphics[width=\textwidth]{./fig/v2}
  \caption{
    \label{fig:spectra}
  }
\end{figure*}

\begin{figure*}[t]
  \includegraphics[width=\textwidth]{./fig/v3}
  \caption{
    \label{fig:spectra}
  }
\end{figure*}

\begin{figure*}[t]
  \includegraphics[width=\textwidth]{./fig/hbt}
  \caption{
    \label{fig:spectra}
  }
\end{figure*}


\section{Acknowledgements}

\medskip
JSM acknowledges support by the DOE/NNSA Stockpile Stewardship Graduate Fellowship under grant no.~DE-FC52-08NA28752.

%\bibliography{trento,duke-qcd-refs/Duke_QCD_refs}


\end{document}
