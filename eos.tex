\documentclass[aps,prc,reprint,amsmath,nofootinbib,superscriptaddress]{revtex4-1}

\usepackage{hyperref}
\usepackage{graphicx}
\usepackage{amsmath}
\usepackage{mathtools}
\usepackage{paralist}
\graphicspath{{fig/}}

%packages for editing
\usepackage{xcolor}
\usepackage[normalem]{ulem}
\makeatletter
\def\uwave{\bgroup \markoverwith{\lower3.5\p@\hbox{\sixly \textcolor{red}{\char58}}}\ULon}
\font\sixly=lasy6 % does not re-load if already loaded, so no memory problem.
\makeatother
\usepackage{mdwlist}

\newcommand{\nch}{N_\text{ch}}
\newcommand{\hotqcd}{HQCD~}
\newcommand{\wb}{WB~}
\newcommand{\spv}{S95~}

\begin{document}

\title{Hydrodynamic simulations of relativistic heavy-ion collisions\\ with different calculations of the QCD equation of state}

\author{J.\ Scott Moreland}
\affiliation{Department of Physics, Duke University, Durham, NC 27708-0305}
\author{Ron A.\ Soltz}
\affiliation{Lawrence Livermore National Laboratory, Livermore, CA 94551-0808}

\date{\today}

\begin{abstract}
Hydrodynamic calculations of ultra-relativistic heavy ion collisions are performed using the iEBE-VISHNU 2+1D code with fluctuating initial conditions and three different parameterizations of the Lattice QCD equations of state: continuum extrapolations for stout and HISQ/tree actions, as well as the s95p-v1 parameterization based upon calculations on $(32^3 \times 8)$ lattices using the p4 action.  All parameterizations are matched to a hadron resonance gas equation of state at $T=$155~MeV, at which point the calculations are continued using the UrQMD hadronic cascade. Final state particle spectra for pions, kaons, and protons and HBT pion radii are compared for each of the equations of state as well as to experimental data for $\sqrt{s_{NN}}=200$ GeV Au+Au collisions with three centrality classes. The differential second ($v_2$) and third ($v_3$) order hydrodynamic flow coefficients are also examined at the end of the hydrodynamic stage. Experimental observables for the stout and HISQ/tree equations of state are observed to differ by less than a few percent for all observables, while the s95p-v1 equation of state generates spectra and flow coefficients which differ by $\sim$10--20\%. As a final check, we evolve a sample of equation of state curves drawn from the HISQ/tree published errors to examine the effect on mean $p_T$ and anisotropic flow coefficients $v_2$ and $v_3$. We observe small (order 1\%) deviations in these observales which indicates that errors in the continuum extrapolation are well constrained. 
\end{abstract}

\maketitle

\section{Introduction}

Quantum Chromodynamics (QCD) predicts that at sufficiently high temperature or density, nuclear matter exists in a state of deconfined state of quarks and gluons known as a quark-gluon plasma (QGP). 
This state of matter filled the early universe several microseconds after the big bang and is now recreated and studied in the laboratory by colliding heavy ions at relativistic energies at the Relativistic Heavy Ion Collider (RHIC) and the Large Hadron Collider (LHC).

Quantitative model to data comparison, using simulations based on relativistic hydrodynamics, is the primary means to extract properties of QGP produced by relativistic heavy-ion collisions which expands and freezes into hadrons too quickly for direct observation. 
These hydroynamic descriptions require two essential ingredients to specify the full time evolution of the QGP fireball: initial conditions which describe the thermal profile of the QGP droplet at some early starting time and a QCD Equation of State (EoS) which interrelates the energy density, pressure and temperature of each fluid cell in local thermal equillibrium.

Lattice discretization is the only reliable method to calculate the QCD equation of state in the vicinity of the QGP phase transition and hence constitutes a critical component of hydrodynamic simulations. 
While lattice techniques are rigorous in their treatment of the underlying QCD Lagrangian, they are subject to statistical and systematic errors inherent in the lattice discretization procedure. 
These errors are manifest in differences in the continuum extrapolated QCD trace anomaly and lead to an overall uncertainty in the true value of the QCD equation of state at zero baryochemical potential.

To date there have been few sensitivity studies on the influence of the EoS on hydroynamic simulation results. 
They have been limited to studies of the order of the phase transition~\cite{Huovinen:2005gy}, different parameterization schemes for the LQCD EoS \cite{Huovinen:2009yb} and data driven Bayesian techniques to constrain parameterizations of the EoS motivated by LQCD calculations \cite{Pratt:2015zsa}.
However, a sensitivity study on the inherent errors in the LQCD EoS has not yet been performed, primarily because continuum extrapolations for the LQCD EoS at zero baryon density have only recently become available \cite{Borsanyi:2013bia,Bazavov:2014pvz}. 
In this work, we quantify the effect of lattice errors on simulations of relativistic heavy-ion collisions by comparing simulation predictions obtained with QCD EoS calculations by the Wuppertal-Budapest collaboration using the stout fermion action \cite{Borsanyi:2013bia} and the HotQCD collaboration using the HISQ/tree action \cite{Bazavov:2014pvz}.  
We also compare to the older s95p-v1 parameterization \cite{Huovinen:2009yb} constructed from calculations performed on coarser ($32^3 \times 8$) lattices using p4 and asqtad actions without contiuum extrapolation.  
The equations of state are analyzed using a modern event-by-event hybrid simulation which couples viscous hydrodynamics to a hadronic afterburner to calculate flows, spectra and Bertsch-Pratt radii and are compared to measurements at the Relativistic Heavy-Ion Collider (RHIC).
We also perform a set of calculations in which the HISQ/tree continuum EoS is sampled from within the published error range.


\section{Equations of State}
\label{eos}

The Wuppertal-Budapest, HotQCD and s95p-v1 EoS parameterizations used in this work all employ staggered fermion actions with varying level improvements: additional terms added to remove lattice artifacts and improve simulation convergence. 
For example, both the stout and HISQ/tree actions used by the Wuppertal-Budapest and HotQCD calculations contain additional smearing of the gluon links relative to p4 and asqtad actions used to construct the s95p-v1 parameterization.
Moreover, the Wuppertal-Budapest stout action omits second order corrections in the lattice spacing which are common to the other three. 

The three analyses are further distinguished by the granularity of the lattices used in each calculation.
The p4 and asqtad results used in the s95p-v1 parameterization are from ($32^3 \times 8$) lattices, referred to by the number of temperal dimension, $N_{\tau}=8$, while the HISQ/tree continuum extrapolation was calculated for $N_{\tau}=8$, 10, and 12, and the stout results for lattices with $N_{\tau}=6$, 8, 10 and 12. 
For a more detailed discussion of the EoS calculations and relative improvements of the staggered fermion actions see \cite{Soltz:2015ula}.

LQCD EoS calculations are obtained from the trace of the stress-energy tensor, equal to the difference between the energy density and three times the pressure. 
This quantity is typically referred to as the interaction measure or trace anomaly  because it measures deviations from the conformal equation of state. 
Once scaled by powers of the temperature, the trace anomaly forms a dimensionless measure 
\begin{equation}
 I \equiv \frac{\Theta^{\mu\mu}(T)}{T^4} = \frac{e - 3p}{T^4},
\end{equation}
where $\Theta$ is the stress-energy tensor, $e$ is the local fluid energy density, $p$ the pressure and $T$ the temperature.

Lattice calcuations typically extend down to temperatures of ${\sim}130$~MeV, where small deviations with the Hadron Resonance Gas (HRG) EoS may begin to develop.
This is evident in Fig.~\ref{fig:trace}, which shows the trace anomaly of the HRG EoS alongside results from the HotQCD and Wuppertal-Budapest (WB) collaborations with HISQ/tree and stout actions respectively, as well as the older s95p-v1 parameterization obtained using p4 and asqad actions.  
Both the HISQ/tree and stout EoS results begin to pull away from the HRG EoS at temperatures above $130$~MeV, while the s95p-v1 parameterization agrees with the HRG results up to a matching temperature of $183.8$~MeV by construction.

Although both the Wuppertal-Budapest and HotQCD collaborations have provided parameterizations suitable for insertion into hydrodynamic codes, the matching temperature of 130~MeV falls below the 155--165~MeV temperature range where hybrid simulations typically switch from relativistic viscous hydrodynamcs to a microscopic kinetic description such as UrQMD~\cite{Bass:1998ca,Bleicher:1999xi}.  
We note also that recent estimates for the freeze-out temperature derived from combining lattice calculations and experimental data also fall within this range~\cite{Bazavov:2014xya, Adare:2015aqk}.  
To ensure a self consistent description of the collision dynamics where the simulation switches from hydrodynamics to microscopic transport, we modify each lattice EoS to match the HRG EoS at the desired hydro-to-micro switching temperature. We thus define a new piecewise interaction measure
\begin{equation}
 \label{interaction}
 I(T) =
  \begin{cases}
   I_\text{hrg}(T)	& T \le T_1, \\
   I_\text{blend}(T)	& T_1 < T < T_2, \\ 
   I_\text{lattice}(T)	& T \ge T_2,
  \end{cases}
\end{equation}
where $I_\text{hrg}$ and $I_\text{lattice}$ are the HRG and LQCD trace anomalies pictured in Fig.~\ref{fig:trace}, and $I_\text{blend}$ is a function 
\begin{equation}
  \label{interpolation}
  I_\text{blend} = (1-z)\, I_\text{hrg} + z\, I_\text{lattice}
\end{equation}
which smoothly connects between the two in the temperature interval $T_1 < T < T_2$. The interpolation parameter $z \in [0,1]$ is constructed to match the first and second derviatives at the endpoints of the interpolation interval,
\begin{eqnarray}
 \label{smoothstep}
 \cr z &=& 6 x^5 - 15 x^4 + 10 x^3 \\
  \text{where } x &=& (T - T_1)/(T_2 - T_1).
\end{eqnarray}
We fix the boundaries of the blending region $T_1=155$~MeV and $T_2=180$~MeV to impose matching at the switching temperature $T_\text{sw} = 155$~MeV which coincides with the pseudo-critical phase transition temperatures of the HotQCD and Wuppertal-Budapest EoS. The modified interaction measures, hereafter referred to simply as HQCD, WB and S95, are plotted in Fig.~\ref{fig:trace_final}\,. We see that the interpolation procedure imposes the necessary matching condition on either side of the switching temperature (vertical gray line) with minimal disturbance to the peak of the LQCD trace anomaly at higher temperatures.

Signal propagation in the QGP medium is characterized by the speed of sound which is expressed in terms of the pressure and energy density as $c_s^2 = dp/de$.
In Fig.~\ref{fig:cs} we plot the squared speed of sound for the HQCD, WB and S95 interaction measures shown in Fig.~\ref{fig:trace_final} alongside recent results from a systematic Bayesian analysis used to constrain parameterized forms of the LQCD EoS by simultaneously fitting model predictions to multiple observables at RHIC and the LHC \cite{Pratt:2015zsa}. 
The top panel of Fig.~\ref{fig:cs} shows the three lattice parameterizations used in this work plotted against 50 parametric EoS samples (thin grey lines) from the Bayesian prior, while the bottom panel of Fig.~\ref{fig:cs} shows the same lattice results plotted against samples from the Bayesian posterior, i.e. once the EoS curves have been constrained by data. The more tightly clustered posterior curves show a clear preference for the present lattice results, although these constraints are not able to resolve differences between the different lattice calculations.

Within the three lattice calculations used in this study, the \hotqcd and \wb speed of sound curves are in good agreement while the \spv parameterization remains softer in a wider interval about the QGP phase transition. We note that the parametric transition \eqref{interpolation} modifies the speed of sound in the vicinity of the EoS matching temperature (vertical gray line) but is constructed to preserve continuity across the desired transition region.

\begin{figure}[t]
  \includegraphics[width=\columnwidth]{./fig/trace}
  \caption{\label{fig:trace} The temperature scaled QCD interaction measure for a hadron resonance gas (HRG) alongside recent lattice calculations from the HotQCD (HQCD) and Wuppertal-Budapest (WB) collaborations as well as the older s95p-v1 lattice parameterization \cite{Bazavov:2014pvz, Borsanyi:2013bia, Huovinen:2009yb}.}
\end{figure}

\begin{figure}[b]
  \includegraphics[width=\columnwidth]{./fig/trace_matched}
  \caption{\label{fig:trace_final} The modified QCD interaction measure for the \hotqcd, \wb and \spv EoS obtained from equation \eqref{interaction} and the corresponding lattice
	  parameterizations in Fig.~\ref{fig:trace}. The vertical gray line marks the hydro-to-micro switching temperature $T_\text{sw} = 155$~MeV.}
\end{figure}

With the trace anomalies in hand, the energy density, pressure and entropy density are easily interrelated to specify the equation of state used in the analysis,
\begin{eqnarray}
 \label{conversion}
 \cr \frac{p(T)}{T^4} &=& \int\limits_0^T dT'\, \frac{I(T')}{T'}, \\
 \frac{e(T)}{T^4} &=& I(T) + 3\, \frac{p(T)}{T^4}, \\
 \frac{s(T)}{T^3} &=& \frac{e(T) + p(T)}{T^4}. 
\end{eqnarray}
 
For clarity, Figs.~\ref{fig:trace}--\ref{fig:cs} do not include the respective errors bands for the HotQCD and Wuppertal-Budapest trace anomalies, but both calculations devote considerable effort
to providing an accurate error estimate for their respective calculations~\cite{Borsanyi:2013bia,Bazavov:2014pvz}.  Common contributions to the errors
come from variations in spline fits to the interaction measures, differences between quadratic and quartic extrapolations in the lattice spacing, and
small (~2\%) variations in the temperature scale.  Errors are typical of order 5\% for most quantities, and increase to 5--10\% in the transition region
where the curves are steepest.

%RAS-end-comment-region

\section{Hybrid Model}

The equations of state are embedded in the modern event-by-event iEBE-VISHNU hybrid model which uses the VISH2+1 boost-invariant viscous hydrodynamics code to simulate the 
time evolution of the QGP medium and a microscopic UrQMD hadronic afterburner for subsequent evolution below the QGP transition temperature. Where necessary, free
parameters of the model are are tuned to facillitate model-to-data comparison with $200$ GeV gold-gold collisions at RHIC. In this section, we briefly outline
the implementation of the model used in the analysis; for a more detailed explanation of the model see reference \cite{Shen:2014vra}. 

\subsection{Initial conditions}
\label{initial_condition}

The initial conditions represent the largest source of uncertainty in modern hydrodynamic simulations and a number of models exist in the literature which have described
the experimental data with varying degrees of success \cite{Schenke:2012wb, Niemi:2015qia, Chatterjee:2015aja, Moreland:2014oya, Drescher:2006pi, Adler:2013aqf}. Since the present work is primarily interested in measuring the sensitivity of the hydrodynamic evolution to differences in the QGP EoS and \emph{not} obtaining the overall best fit of model to data, we choose perhaps the simplest and most widely adopted initial condition implementation based on a two-component Glauber model; for an overview see \cite{Miller:2007ri}.

In the two-component ansatz, initial entropy is deposited proportional to a linear combination of nucleon participants and binary nucleon-nucleon collisions,
\begin{equation}
 dS/dy \,\vert_{y=0} \propto \frac{(1-\alpha)}{2}N_\text{part} + \alpha N_\text{coll}
 \label{twocomponent}
\end{equation}
where for the binary collision fraction, we use $\alpha=0.14$ which has been shown to provide a good description of the centrality dependence of charged particle 
multiplicity in $200$ GeV gold-gold collisions \cite{Shen:2014sfi}.

The entropy is localized about each nucleon's transverse parton density $T_p({\bf x})$,
\begin{eqnarray}
 dS/dy \,\vert_{y=0} &\propto& \smashoperator{\sum_{i=0}^{N_\text{part,A}}} w_i\, T_p({\bf x} - {\bf x}_i)(1-\alpha + \alpha\, N_\text{coll,i}) \nonumber \\
                     &+& \smashoperator{\sum_{j=0}^{N_\text{part,B}}} w_j\, T_p({\bf x} - {\bf x}_i)(1-\alpha + \alpha\, N_\text{coll,j})
 \label{glauber}
\end{eqnarray}
where the summations run over the participants in each nucleus, $N_\text{coll,i}$ denotes the number of binary collisions suffered by the $i$-th nucleon 
and the proton density $T_p({\bf x})$ is described by a Gaussian
\begin{equation}
 T_p({\bf x}) = \frac{1}{\sqrt{2 \pi B}} \exp \left(-\frac{x^2+y^2}{2 B} \right)
\end{equation}
with transverse area $B = 0.36$ $\text{fm}^2$.

The random nucleon weights $w_i$ in equation \eqref{glauber} are sampled independently from a Gamma distribution with unit mean
\begin{equation}
 P_k(w) = \frac{k^k}{\Gamma(k)} w^{k-1} e^{-k w},
\end{equation}
and shape parameter $k = \text{Var}(P)^{-1}$ which modulates the variance of the distribution. 
Such fluctuations are typically added to reproduce the large multiplicity fluctuations observed in minimum bias proton-proton collisions \cite{Adare:2008ns, Dumitru:2012yr, Moreland:2012qw, Bozek:2013uha, Shen:2014sfi}. 
In this work the shape parameter is fixed to $k=1$ determined by a fit to the $200$ GeV UA5 data \cite{Ansorge:1988kn}. 

The initial condition profiles, which provide the entropy density $dS/(d^2r_\perp\, d\eta\, \tau_\text{therm})$ at the QGP thermalization time, are finally 
rescaled by an overall normalization factor to fit the measured charged particle multiplicity in $0$--$10\%$ centrality collisions.

\begin{figure}
  \includegraphics[width=\columnwidth]{./fig/cs}
  \caption{\label{fig:cs} Squared speed of sound $c_s^2$ plotted versus temperature $T$ for the \hotqcd (green), \wb (purple) and \spv (blue) equations of state pictured in 
           Fig.~\ref{fig:trace_final}. Top and bottom panels show EoS parameterizations from the Bayesian analysis in reference \cite{Pratt:2015zsa} for the prior (top panel) and posterior 
           (bottom panel) superimposed as thin grey lines. The vertical gray line indicates the switching temperature $T_\text{sw} = 155$~MeV where the model switches from fluid 
           dynamics to microscopic transport in the present study.}
\end{figure}

\begin{figure*}[t]
  \includegraphics[width=\textwidth]{./fig/spectra}
  \caption{
    \label{fig:spectra} Effect of the equation of state on transverse momentum spectra. Top row: model calculations using the \hotqcd equation of state plotted against 
    PHENIX data for pions, kaons and protons (blue lines/circles, red lines/squares and green lines/triangles) in centrality bins $0$--$5\%$, $20$--$30\%$ and $40$--$50\%$ 
    (columns left to right). Middle and bottom rows: ratios of the \wb and \spv invariant yields to the \hotqcd result. Shaded bands indicate two sigma statistical error. }
\end{figure*}

\begin{figure*}[t]
  \includegraphics[width=\textwidth]{./fig/v2}
  \caption{
    \label{fig:v2} Effect of the equation of state on differential elliptic flow $v_2(p_T)$ calculated from the Cooper-Frye freezeout hypersurface \eqref{differential_flow}.
    Top row: model calculations using the \hotqcd equation of state for the elliptic flow $v_2(p_T)$  of pions, kaons and protons (blue, orange and green lines) 
    in centrality bins $0$--$10\%$, $20$--$30\%$ and $40$--$50\%$ (columns left to right). Middle and bottom rows: ratios of the \wb and \spv elliptic flow to 
    the \hotqcd result. Statistical errors are smaller than the linewidth and have been omitted.
  }
\end{figure*}

\begin{figure*}[t]
  \includegraphics[width=\textwidth]{./fig/v3}
  \caption{
    \label{fig:v3} Same as Fig.~\ref{fig:v2} but for differential triangular flow $v_3(p_T)$. Note that the y-axis limits in the top row are different.
  }
\end{figure*}

\begin{figure*}[t]
  \includegraphics[width=\textwidth]{./fig/hbt}
  \caption{
    \label{fig:hbt} Effect of the equation of state on the Bertsch-Pratt radii.  We plot $R_s$, $R_o$, $R_l$ and the ratio $R_o/R_s$ (rows top to bottom) 
    in centrality bins $0$--$10\%$, $20$--$30\%$ and $40$--$50\%$ (columns left to right) against transverse mass $m_T$ for the \hotqcd, \wb  and \spv equations of state 
    (purple, orange and blue lines). Shaded bands indicate two sigma errors estimated from the Jacobian of the fit function \eqref{fitfunction}. Symbols with errors bars 
    are experimental data from PHENIX.  
  }
\end{figure*}


\subsection{Hydrodyamics and Boltzmann transport}

The hydrodynamic equations of motion are obtained in the iEBE-VISHNU model by solving the second-order Israel-Stewart equations,
\begin{equation}
 \partial_\mu T^{\mu\nu} = 0, \quad T^{\mu\nu} = e u^\mu u^\nu - (p + \Pi) \Delta^{\mu\nu} + \pi^{\mu\nu},
\end{equation}
where the bulk pressure $\Pi$ and shear stress $\pi^{\mu\nu}$ satisfy the relaxation equations,
\begin{eqnarray}
 \label{viscosity}
 \cr \mathcal{D}\Pi = &-&\frac{1}{\tau_\Pi}(\Pi + \zeta \theta) - \frac{1}{2} \Pi \frac{\zeta T}{\tau_\Pi}d_\lambda \left(\frac{\tau_\Pi}{\zeta T} u^\lambda \right), \nonumber \\
  \Delta^{\mu\alpha} \Delta^{\nu\beta} \mathcal{D}\pi_{\alpha\beta} = &-&\frac{1}{\tau_\pi}(\pi^{\mu\nu} - 2 \eta \sigma^{\mu\nu}) \\
  &-& \frac{1}{2} \pi^{\mu\nu} \frac{\eta T}{\tau_\pi} d_\lambda\left(\frac{\tau_\pi}{\eta T} u^\lambda \right ).
\end{eqnarray}

We follow the work in reference \cite{Shen:2014sfi} and fix the bulk viscosity $\zeta$ and shear viscosity $\eta$ in equation \eqref{viscosity} using a constant specific shear viscosity $\eta/s=0.08$ 
and vanishing bulk viscosity $\zeta/s=0$ in the hydrodynamic phase of the simulation. It would be interesting to study the effect of bulk viscous corrections which are sensitive to 
the peak of the QCD trace anomaly near the QGP phase transition \cite{Karsch:2007jc}. Unfortunately, bulk viscous corrections to particle emission at freezeout do not have a straight forward implementation in the present hybrid model 
and are neglected in this work. 

As previously explained in section \,\ref{eos}, the iEBE-VISHNU hybrid model transitions from hydrodynamic field equations to microscopic transport at a sudden switching temperature $T_\text{sw}$ 
at which the hydrodynamic energy-momentum tensor is particlized using the Cooper-Frye freezeout prescription,
\begin{equation}
 E\frac{dN_i}{d^3p} = \int_\sigma f_i(x,p) p^\mu d^3\sigma_\mu
 \label{cooper-frye}
\end{equation}
where $f_i$ is the distribution function of particle species $i$, $p^\mu$ is its four-momentum and $d^3\sigma_\mu$ characterizes an element of the isothermal freezeout 
hypersurface defined by $T_\text{sw}$.

The sampled particles then enter the UrQMD simulation where the Boltzmann equation, 
\begin{equation}
 \frac{df_i(x,p)}{dt} = \mathcal{C}_i(x,p),
\end{equation}
is solved to simulate all elastic and inelastic collisions between the particles with collision kernel $\mathcal{C}_i$ until the system becomes too dillute to continue interacting. 
Finally, the four-position, four-momentum and particle identification number of each particle is recorded. 

\section{Results}
\label{results}

The results section is organized as follows. In sub-section \ref{spectra} we calculate the particle spectra for each equation of state across three different centrality classes using the final 
particle information output by the hybrid simulation. In sub-section \ref{flow} we repeat the calculation for elliptic and triangular flow but perform the calculation on the 
hydrodynamic Cooper-Frye freezeout surface for reasons explained later in the text. In sub-section \ref{hbt} we calculate the femptoscopic event-averaged Bertsch-Pratt radii, again using the 
final particle information output by the full hybrid calculation. Finally in sub-section \ref{errors}\,, we calculate mean $p_T$ and integrated anisotropic flow cumulants $v_2\{2\}$ and $v_3\{3\}$ from the UrQMD output using a sampling of equation of state curves from the HotQCD published errors. 

All results presented in the following sections are based on $5\cdot10^4$ minimum bias events which are subdivided into centrality classes according to initial entropy, e.g. the initial condition events with $20\%$ highest entropy comprise centrality class $0$--$20\%$. Each hydrodynamic event is oversampled an additional ten times to increase the number of particles in each event and suppress finite statistical error.

\subsection{Particle spectra}
\label{spectra}

Figure \ref{fig:spectra} shows the invariant yield $dN/(2\pi p_T dp_T dy)$ of positively charged pions, kaons and protons calculated from the hybid model for the $0$--$5\%$, $20$--$30\%$
and $40$--$50\%$ centrality classes using the \hotqcd, \wb and \spv equations of state constructed in section \ref{eos}. 

The first row shows the \hotqcd yields obtained from the hybrid model plotted against observed pion, proton and kaon data from PHENIX. The second and third rows show the ratio of the invariant yields of the \wb and \spv equations of state over the the \hotqcd result. One sees that the \hotqcd equation of state provides a good description of observed particle yields except for at moderate to large $p_T$ in central collisions where the equation of state overpredicts the data. This agreement would likely improve with more realistic initial conditions, bulk viscous corrections and/or more careful treatment of the hydro-to-micro switching temperature $T_\text{sw}$, and thus it's difficult to make any specific statements on the overall fit of model to data. It suffices to say that the most recent HotQCD lattice results provide a reasonable description of the PHENIX data and agrees within the overall uncertainty of the present model. 

The second and third rows of Fig.~\ref{fig:spectra} show the ratios of the \wb and \spv yields to the \hotqcd result. The observed spectra predicted by the \hotqcd and \wb equations of state agree within statistical error, while the \spv equation of state is appreciably softer and produces $\sim\!5\%$ more particles at $p_T = 0.5$ GeV and $\sim\!20\%$ fewer particles at $p_T=2.5$ GeV across all three centralities.

\subsection{Elliptic and triangular flows}
\label{flow}

The azimuthal anisotropy of final particle emission is characterized by the Fourier expansion
\begin{equation}
 E \frac{d^3N}{d^3p} = \frac{1}{2\pi} \frac{d^2N}{dy p_T dp_T} \left(1 + \sum\limits_{n=1}^\infty 2 v_n \cos n(\phi - \Psi_{RP}) \right)
\end{equation}
where $\phi$ is the direction of the emitted particle, $\Psi_{RP}$ is the reaction plane angle of the event and $v_n$ the anisotropic flow coefficient corresponding to the Fourier harmonic of order $n$.

The reaction plane angle cannot be measured experimentally and the anisotropic flow is typically estimated using multi-particle correlations such as two and four-particle cumulants. The statistical error of the event-averaged estimators is suppressed with both increasing event multiplicity and event sample size. This can pose a challenge for computationally intensive hybrid model calculations which typically cannot reach integrated luminosities comparable to experiment. 

Statistical errors are particularly noxious in differential flow calculations at moderate to large $p_T$ where particle statistics are limited. We circumvent this issue in the differential flow analysis and calculate the flow anisotropy of pions, kaons and protons directly from the Cooper-Frye freezeout surface using the built in routines in the iEBE-VISHNU package according to
\begin{equation}
 \label{differential_flow}
 v_n(p_T) = \frac{\int d\phi_p e^{i n \phi_p} dN/(dy p_T dp_T d\phi_p)}{\int d\phi_p\, dN/(dy p_T dp_T d\phi_p)}.
\end{equation}
Consequently, the present flow sensitivity analysis does not include contributions from flow generated by the UrQMD hadronic afterburner which is indentical for each of the three equations of state. Hence, the following
results should be interpretted as a conservative \emph{upper} bound on the goodness of fit sensitivity expected in a full hybrid model simulation.

Figs.~\ref{fig:v2} shows the elliptic flow $v_2$ of pions, kaons and protons calculated from equation \eqref{differential_flow} for the \hotqcd, \wb and \spv equations of state in $0$--$10$, $20$--$30$ and $40$--$50\%$ centrality bins. 
The first row of the figure shows the elliptic flow predicted by the \hotqcd equation of state while the middle and bottom rows display theoretical ratios of the \wb and \spv predictions over the \hotqcd result. The information
in Fig.~\ref{fig:v3} is identical to that in Fig.~\ref{fig:v2} except that elliptic flow $v_2$ has been replaced with triangular flow $v_3$.

We see in Fig.~\ref{fig:v2} that the elliptic flow generated by the \hotqcd and \wb parameterizations is in very good agreement across all centralities, while the \spv parameterization systematically generates $\sim \! 5\%$ less flow than the \hotqcd equation of state. This is expected as the \spv equation of state is considerably softer in the vicinity of the phase transition as evidenced by the speed of sound in Fig.~\ref{fig:cs}. 

In Fig.~\ref{fig:v3}, we see that the effect on the triangular flow is similar to the effect observed on the elliptic flow except more pronounced and generates as large as a $15\%$ discrepancy in the peripheral flows predicted by the \hotqcd and \spv equations of state. This sensitivity of higher harmonics to the softness of the QGP phase equation of state puts the relatively large higher-order anisotropic flow coefficients observed at RHIC and the LHC into perspective. 

\subsection{Femptoscopic Bertsch-Pratt radii}
\label{hbt}

The size of the fireball emission region is estimated using Hanbury-Brown-Twiss (HBT) interferometry for identical particles. The azimuthally averaged two-particle correlation function 
\begin{equation}
 \label{hbt}
 C(q, k) = \frac{\sum\limits_n \sum\limits_{i, j} \delta_{q} \, \delta_{k}\Psi(q,r)}{\sum\limits_{n} \sum\limits_{i,j'} \delta_{q} \, \delta_{k}}
\end{equation}
consists of a numerator with particles pairs sampled from the same event and a denominator with pairs sampled from different events. Here $q = p_i - p_j$ denotes the relative momentum, $r=x_i-x_j$ the relative separation and $k = (p_i + p_j)/2$ the average momentum of the pion pair in the longitudinal co-moving 
frame where the compononent of $k$ along the beam axis vanishes. The numerator is summed over all events $n$ in a given centrality class and unique particle pair combinations $i,j$ in each event. In the denominator, particle $i$ is
taken from one event and particle $j'$ from a random partner event in the same centrality class. The delta functions $\delta_q$ and $\delta_k$ are $1$ if the momenta $q$ and $k$ fall into their respective bins and $0$ otherwise. Bose-Einsten 
correlations, which are not included natively in the UrQMD model, are imposed by adding the symmetrization factor $\Psi(q,r) = 1 + \cos q\,r$. 

The average pair momentum $k$ is then projected into its longitudinal component $k_\text{z}$ and transverse component $k_T$, while the separation momentum $q$ is represented in the orthogonal coordinates 
$(q_o, q_s, q_l)$, where $q_l$ lies along the beam axis, $q_o$ is parallel to $k_T$ and $q_s$ perpendicular to $q_o$ and $q_l$. The resulting correlation function is approximated using a Gaussian source and 
fit to the parametric form
\begin{equation}
 \label{fitfunction}
 C(q_o, q_s, q_l, k_T) = \mathcal{N} \left(1 + \lambda\, e^{-R_o^2 q_o^2 - R_s^2 q_s^2 - R_l^2 q_l^2} \right) 
\end{equation}
by finding the best fit normalization $\mathcal{N}$, source strength $\lambda$ and Bertsch-Pratt radii $R_o$, $R_s$ and $R_l$ for a given transverse momentum $k_T$. 

We calculate the Bertsch-Pratt radii for each equation of state using identical pions. The fit is perfomed using $5\cdot10^4$ hydrodynamic events in each centrality bin and an additional ten UrQMD oversamples per event. The oversamples are then concatenated into a single particle list to increase the number of particle pairs by a factor $10^2$.

In Fig.~\ref{fig:hbt}, we plot the Bertsch-Pratt radii for the \hotqcd, \wb and \spv equations of state as functions of the transverse mass $m_T = \sqrt{m^2 + k_T^2}$ where $m$ is the pion mass. The horizontal rows show the radii $R_s$, $R_o$, $R_l$ and ratio $R_o/R_s$ (top to bottom), while the columns mark centrality classes $0$--$10\%$, $10$--$20\%$ and $20$--$40\%$ (left to right). The different colored lines annotated in the legend indicate different equations of state and the bands estimate the error from the Jacobian 
of the fit. The symbols with error bars are experimental data from PHENIX.

We see that the Bertsch-Pratt radii predicted by the hybrid model provide a good description of the data across all centralities, except for at low $p_T$ where the ratio $R_o/R_s$ slightly undershoots the data. However, in contrast to spectra and flows we see no discernible difference in the Bertsch-Pratt radii predicted by the three different equations of state. This suggests that femptoscopic measurements are not sensitive enough to resolve small differences in the lattice equation of state.

\begin{figure}[t]
  \includegraphics[width=\columnwidth]{./fig/splines}
  \caption{
    \label{fig:splines}
    Temperatured scaled interaction measure $(e-3 p)/T^4$ as a function of temperature $T$ for 100 different samples of the HotQCD error estimate (thin blue lines) plotted alongside the HRG EoS (thick orange line) and best fit HotQCD parameterization (thick red line).
  }
\end{figure}

\begin{figure}[t]
  \includegraphics[width=\columnwidth]{./fig/eos_compare}
  \caption{
    \label{fig:splines}
    Pion, kaon and proton yields in centrality bin 19--30\% for 100 random samples drawn from the HotQCD error splines. Top panel shows the invariant yield plotted against PHENIX data. Bottom panels show the ratio of each spline
    to the published HotQCD EoS.
  }
\end{figure}

\subsection{HotQCD errors}
\label{errors}

In addition to the best fit parameterization shown in Fig.~\ref{fig:trace}~, the HotQCD analysis includes a detailed calculation of associated lattice errors. The errors enter the analysis in several steps.
The HotQCD trace anomaly is first calculated at various temperatures in the interval $130 < T < 400$ MeV using grids with temperal extent $N_\tau = 8,10$ and $12$.
For each temperature and grid size, the lattice simulation is repeated a large number of times. 
This creates a set of ``data points'' with a mean and variance determined from the Monte Carlo ensemble. 
Each data point is then resampled from the ensemble's mean and variance, and the collection of resampled points, one for each value of the temperature $T$ and grid size $N_\tau$, is fit with the ansatz,
\begin{equation}
 \label{splines}
 \frac{\theta^{\mu\mu}(T)}{T^4} = A + \sum\limits_{i=1}^{n_k=3} B_i \times S_i(T) + \frac{C + \sum_{i=1}^{n_k + 3} D_i \times S_i(T)}{N_\tau^2}.
\end{equation}
Here the constants $A$, $B_i$, $C$ and $D_i$ are parameters of the fit, $S_i$ is a set of cubic basis splines and $n_k$ the number of knots used in the B-spline fitting. 
The entire procedure is then repeated 20,001 times to sample the function space of $\theta^{\mu\mu}(T)/T^4$ from the errors in the ensemble averaged lattice measurements.

In this section, we investigate the effect of the HotQCD lattice errors by measuring the spectra and flows for a subset of 100 randomly sampled EoS curves determined according to equation \eqref{splines}.
The piecewise interpolation procedure described in section \ref{eos} is applied to each spline to smoothly match the HotQCD lattice interaction measures with the HRG result at low temperature. 
The resulting interaction measures are shown in Fig.~\ref{fig:splines} alongside the best fit HotQCD parameterization which naturally falls in the middle of the sampled curves. 
The vertical spread of the curves should be interpretted as the one sigma error band in the interval 180--400 MeV. 

The energy density, entropy density, pressure and temperature are then calculated from each interaction measure according to \eqref{conversion} to generate 100 different EoS tables. 
Above 400~MeV, the higher derivatives of the interaction measures become unreliable and we extrapolate the EoS table using a simple power law. 
We note, however, that this modification has negligible impact on the hydrodynamic evolution at RHIC where the system is predominantly below 400~MeV.

The HotQCD EoS samples are used to repeat the spectra and flow analysis presented in section \ref{results}. In Fig.~\ref{fig:spectra} we show the invariant yield $dN/(2\pi p_T dp_T dy)$ of positively charged pions, kaons and protons
in centrality bins $0$--$5\%$, $20$--$30\%$ and $40$--$50\%$. The ratio plot in the figure subset shows that the spread in the particle yield for the HotQCD errors is ${\sim} 5\%$ across all centralities. 
This spread is significantly smaller than the ${\sim}20\%$ deviations observed between the different EoS in Fig.~\ref{fig:spectra}. The elliptic flow $v_2$ shown in Fig.~\ref{fig:v2} and triangular flow $v_3$ in Fig.~\ref{fig:v3}
show similarly small sensitivies to the HotQCD errors with variations of ${\sim}5\%$ for $v_2$ and ${\sim}10\%$ for $v_3$. This establishes that the error in the HotQCD calculation leads to minor variations of hydrodynamic
observables.

\section{Conclusion and Outlook}
\label{conclusion}

The LQCD EoS is an essential ingredient used in hydrodynamic simulations of relativistic heavy-ion collisions. In this study we simulated collisions at RHIC using a modern event-by-event hybrid model with three different equations of state: 
two state of the art calculations by the HotQCD and Wuppertal-Budapest (WB) collaborations as well the s95p-v1 parameterization based on older lattice data. The model is used to quantify differences in the spectra, anisotropic flow coefficients and 
Bertsch-Pratt radii predicted by the different EoS. 

We find that the particle spectra calculated using the latest HotQCD and WB EoS, constructed from HISQ/tree and stout actions respectively, are indistinguishable within the statistical errors of our analysis, while the s95p-v1 EoS constructed from p4 and asqad actions is substantially softer and generates spectra which deviate by as much as $30\%$ within the applicability of the model. 
Similar sensitivity was observed in the elliptic flow harmonics $v_2$ and $v_3$ calculated on the Cooper-Frye freezeout hypersurface. The elliptic flow produced by the HotQCD and WB EoS is essentially identical, while the softer s95p-v1 parameterization produces ${\sim}10\%$ less elliptic flow $v_2$ across all centralities. The triangular flow $v_3$ was slightly more sensitive to differences in the EoS and revealed a small ${\sim} 5\%$ discrepancy between the HotQCD and WB EoS and much larger ${\sim}15\%$ discrepancies with the s95p-v1 EoS. 
On the other hand, the femptoscopic Bertsch-Pratt radii did not reveal meaningful differences in $R_\text{o}$, $R_\text{s}$ and $R_\text{l}$ within the errors of our analysis. 

As a final check, we calculate the spectra and flows for 100 randomly sampled EoS curves drawn from the HotQCD errors at one standard deviation. The resulting variations in the sampled HotQCD curves are smaller than differences between the WB/HotQCD and s95p-v1 EoS, and amount to 
5\% variations in the spectra and 5\% and 10\% variations in the elliptic flow $v_2$ and triangular flow $v_3$ respectively. 

Collectively, these results signal an important convergence in the LQCD EoS at zero baryochemical potential. Not only do the latest HotQCD and WB lattice calculations agree, but a sampling of the HotQCD errors indicate that numerical uncertainties in the calculations are under
good control. This indicates that QGP parameter extractions via systematic model-to-data comparison are robust to small differences between modern LQCD EoS calculations. In contrast, older lattice calculations with a larger peak in the LQCD trace anomaly lead to measureable differences in the QGP evolution and errode the accuracy of QGP model-to-data parameter extraction.

\begin{acknowledgments}
 JSM acknowledges support by the DOE/NNSA Stockpile Stewardship Graduate Fellowship under grant no.~DE-FC52-08NA28752.
\end{acknowledgments}

\bibliography{eos,duke-qcd-refs/Duke_QCD_refs}


\end{document}
