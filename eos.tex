\documentclass[aps,prc,reprint,amsmath,nofootinbib,superscriptaddress]{revtex4-1}

\usepackage{hyperref}
\usepackage{graphicx}
\usepackage{amsmath}
\graphicspath{{fig/}}

\usepackage{mdwlist}

\newcommand{\nch}{N_\text{ch}}

\begin{document}

\title{Hydrodynamic simulations of relativistic heavy-ion collisions\\ with different calculations of the QCD equation of state}

\author{J.\ Scott Moreland}
\affiliation{Department of Physics, Duke University, Durham, NC 27708-0305}
\author{Ron A.\ Soltz}
\affiliation{Lawrence Livermore National Laboratory, Livermore, CA 94551-0808}

\date{\today}

\begin{abstract} 
   
\end{abstract}

\maketitle

\section{Introduction}

%RAS - remove lame utility statement from first sentence
%Fluid dynamics is a useful framework to study the collective behaviour of hot and dense nuclear matter produced in relativistic heavy-ions. 
Quantum Chromodynamics (QCD) predicts that at sufficiently high temperature or density nuclear matter exists in a state of deconfined 
quarks and gluons known as a quark-gluon plasma (QGP).   This state of matter existed within the universe approximately one microsecond 
after the big bang and is recreated in relativistic heavy ion collisions at the Relativistic Heavy Ion Collider (RHIC) and the Large Hadron Collider (LHC).
Quantitative comparisons to simulations based on relativistic viscous hydrodynamics currently provide the primary means of extracting the
properties of the QGP which expands and freezes into hadrons too quickly for direct observation.

The hydroynamic transport equations require two essential ingredients to specify the full time evolution of the QGP fireball: initial conditions 
which describe the thermal profile of the QGP droplet at some early starting time and a QCD Equation of State (EoS) which interrelates energy density, 
pressure and temperature of each fluid cell in local thermal equillibrium.

Lattice discretization is the only reliable method to calculate the QCD equation of state in the vicinity of the QGP phase transition and hence 
constitutes a critical component of hydrodynamic simulations. While lattice techniques are rigorous in their treatment of the underlying QCD 
Lagrangian, they are subject to statistical and systematic errors inherent in the lattice discretization procedure. These errors are manifest in differences in the 
continuum extrapolated QCD trace anomaly and lead to an overall uncertainty in the true value of the 
QCD equation of state at zero baryochemical potential.

% Simulations using a lattice based equation of state inherrit all forms of numeric and systematic uncertainty associated with the underlying 
% lattice methodology. These modeling uncertainties have been studied both at low temperature, by comparing simulations with a lattice equation of 
% state to results obtained from a hadron resonance gas model \cite{Huovinen:2009yb}, and at high temperature by comparing the effect of different 
% lattice parameterizations of the lattice equation of state on particle spectra and flow \cite{Huovinen:2005gy, Huovinen:2009yb}. 

To date there have been few sensitivity studies of the influence of the EoS on hydroynamics simulaion results.  They have been limited to studies the order 
of the phase transition~ \cite{Huovinen:2005gy}, different parameterization schemes for the LQCD EoS ~\cite{Huovinen:2009yb}, and data driven Bayesian 
techniques to determine parameters for an EoS for a parameterization motivated by previous LQCD EoS calculations~\cite{Pratt:2015vb}.  However, a sensitivity
study on the inherant errors in the LQCD EoS has not yet been performed, primarily because continuum extrapolations for the LQCD EoS at zero
baryond density have only recently become available~\cite{Borsanyi:2013bia,Bazavov:2014pvz}.
In this work, we quantify the effect of lattice errors on simulations of relativistic heavy-ion collisions by comparing simulation predictions obtained 
with QCD EoS calculations by the Wuppertal-Budapest collaboration using the stout fermion action~\cite{Borsanyi:2013bia} and the HotQCD collaboration 
using the HISQ/tree action~\cite{Bazavov:2014pvz}.  We also perform a comparison to older s95p-v1 parameterization ~\cite{Huovinen:2009yb}, derived from 
constrained fits to calculations using coarser ($32^3 \times 8$) lattices using the p4 and asqtad actions without contiuum extrapolation.  The equations of 
state are analyzed using a modern event-by-event hybrid simulation which couples viscous hydrodynamics to a hadronic afterburner to calculate flows, spectra 
and Bertsch-Pratt radii and are compared to measurements at the Relativistic Heavy-Ion Collider (RHIC).  We also perform a set of calculations in which the 
HISQ/tree continuum EoS is sampled from within the published error range.


\section{Equations of State}
\label{eos}

The stout and HISQ/tree actions as well as the p4 and asqtad actions are examples of staggered fermion actions that differ in their level improvements: additional terms 
added to remove lattice artifacts, thereby improving convergence.  For a detailed discussion of these improvements see~\cite{Soltz:2015ur}.  For this comparison we note 
that the stout and HISQ/tree actions contain additional smearing of the gluon links relative to p4 and asqtad, and the stout action does not included corrections to second 
in the lattice spacing that are common to the other three.  The p4 and asqtad results used in the s95p-v1 parameterization are from ($32^3 \times 8$) lattices, referred to by 
the number of temperal dimension, $N_{\tau}=8$.  The HISQ/tree continuum extrapolation was calculated for $N_{\tau}=8$, 10, and 12, whereas the stout results were 
obtained from lattices of $N_{\tau}=6$, 8, 10, and 12.

The hybrid simulation used in this work switches from viscous hydrodynamics to a microscopic kinetic description once the system expands, cools and freezes 
into hadrons. While the QCD equation of state enters the hydrodynamic phase of the simulation as a freely specified function interrelating energy density, 
pressure and temperature, its description in the kinetic phase of the collision is fixed by the finite number of particles and particle resonances included 
in the UrQMD model used for the hadronic phase of the simulation.

As a result, we limit our study to differences in the QCD equation of state \emph{above} the QGP transition temperature where hydrodynamics allows us to 
freely vary its chosen form. We study three different parameterizations for this high temperature dependence -- two state of the art calculations 
in 2+1 flavour QCD from the HotQCD \cite{Bazavov:2014pvz} and Wuppertal-Budapest \cite{Borsanyi:2013bia} collaborations, as well as the older s95p-v1 
parameterization \cite{Huovinen:2009yb} constructed using lattice data measured with a coarser lattice spacing \cite{Bazavov:2009zn}.

The QCD equation of state is frequently characterized by the trace of the energy-momentum tensor, also referred to as the trace anomaly or interaction measure. 
When scaled by powers of the temperature, the trace anomaly forms a dimensionless measure
\begin{equation}
 I \equiv \frac{\Theta^{\mu\mu}(T)}{T^4} = \frac{e - 3p}{T^4},
\end{equation}
where $e$ is the local fluid energy density, $p$ the pressure and $T$ the temperature.

In Fig.~\ref{fig:trace} we plot the temperature scaled interaction measure of each equation of state as well as that of a hadron resonance gas calculated from 
the list of partial resonances included in the UrQMD collision kernel. The s95p-v1 parameterization, which was constructed to interpolate between 
a hadron resonance gas calculation at low temperatures and lattice results at high temperatures, is in good agreement with the UrQMD equation of state while the 
HotQCD and Wuppertal-Budapest results are slightly higher in the vicinity of the phase transition.

To ensure a self consistent description in regions of the collision where the simulation switches from hydrodynamics to Boltzmann transport, we match each high 
temperature lattice equation of state with the low temperature UrQMD equation of state. We thus define a piecewise function for the temperature scaled interaction measure,
\begin{equation}
 \label{interaction}
 I(T) =
  \begin{cases}
   I_\text{hrg}(T)	& T \le T_1 \\
   I_\text{blend}(T)	& T_1 < T < T_2 \\ 
   I_\text{lattice}(T)	& T \ge T_2,
  \end{cases}
\end{equation}
where $I_\text{hrg}$ is the hadron resonance gas trace anomaly in UrQMD pictured in Fig.~\ref{fig:trace}, $I_\text{lattice}$ represents one of the HotQCD, Wuppertal-Budapest or S95p-v1 
parameterizations and $I_\text{blend}$ is a function which smoothly connects between the two in the temperature interval $T_1 < T < T_2$,
\begin{equation}
  \label{interpolation}
  I_\text{blend} = (1-z)\, I_\text{hrg} + z\, I_\text{lattice}.
\end{equation}
The interpolation parameter $z \in [0,1]$ in equation \ref{interpolation} is constructed to match the first and second derviatives at the endpoints of the 
interpolation interval,
\begin{eqnarray}
 \label{smoothstep}
 \cr z &=& 6 x^5 - 15 x^4 + 10 x^3 \\
  \text{where } x &=& (T - T_1)/(T_2 - T_1),
\end{eqnarray}
where the endpoints $T_1 = T_\text{sw}$ and $T_2 = 180$ MeV smoothly interpolate the lattice results into the UrQMD trace anomaly at the desired switching 
temperature $T_\text{sw}$.

\begin{figure}[t]
  \includegraphics[width=\columnwidth]{./fig/trace}
  \caption{\label{fig:trace} The temperature scaled QCD trace anomaly for the UrQMD. HotQCD, WB and s95p-v1 parameterizations as a function of temperature \cite{?}.}
\end{figure}

\begin{figure}[b]
  \includegraphics[width=\columnwidth]{./fig/trace_final}
  \caption{\label{fig:trace_final} The modified QCD trace anomalies HotQCD', WB' and S95' obtained from equation \eqref{interaction} and the corresponding lattice
	  parameterizations in Fig.~\ref{fig:trace}. The gray, vertical line marks the hydro-to-micro switching temperature $T_\text{sw} = 154$ MeV.}
\end{figure}

In principle, the switching temperature could assume any value in a small interval below the equation of state's pseudo-critical transition temperature $T_\text{c}$ 
and should be tuned to fit the relative abundance of pions, protons and kaons measured by experiment. Since we are primarily interested in the sensitivity of the 
simulation to changes in the equation of state with all other quantities held fixed, we fix the transition temperature using the HotQCD chiral transition temperature 
$T_\text{sw} = T_\text{c} = 154$ MeV.
 
The modified interaction measures, labeled with a prime to distinguish them from the raw lattice results, are plotted in Fig.~\ref{fig:trace_final}. The vertical gray line 
marks the hydro-to-micro switching temperature $T_\text{sw}=154$ MeV where the model switches from the VISH2+1 hydrodynamics code to UrQMD. 

In Fig.~\ref{fig:cs} we plot the squared speed of sound $c_s^2 = dp / de$ for each modified interaction measure. The speed of sound of the HotQCD' and WB' equations
of state are in good agreement while the S95' parameterization remains softer in a wider interval about the QGP phase transition. We note that the speed of sound in 
the HotQCD' and WB' parameterizations is clearly affected by the parametric transition \eqref{interpolation} in the vicinity of the switching temperature (vertical gray line), 
but that the imposed matching maintains continuity.

With the trace anomalies in hand, the energy density, pressure and entropy density are easily interrelated to specify the equation of state used in the analysis,
\begin{eqnarray}
 \cr \frac{p(T)}{T^4} &=& \int\limits_0^T dT'\, \frac{I(T')}{T'}, \\
 \frac{e(T)}{T^4} &=& I(T) + 3\, \frac{p(T)}{T^4}, \\
 \frac{s(T)}{T^3} &=& \frac{e(T) + p(T)}{T^4}. 
\end{eqnarray}
 

\section{Hybrid Model}

The equations of state are embedded in the modern event-by-event VISHNU hybrid model which uses the VISH2+1 boost-invariant viscous hydrodynamics code to simulate the 
time evolution of the QGP medium and a microscopic UrQMD hadronic afterburner for subsequent evolution below the QGP transition temperature. Where necessary, free
parameters of the model are are tuned to facillitate model-to-data comparison with $200$ GeV gold-gold collisions at RHIC. In this section, we briefly outline
the implementation of the model used in the analysis; for a more detailed explanation of the VISHNU model, we refer the reader to \cite{}. 

\subsection{Initial conditions}
\label{initial_condition}

The hydrodynamic initial conditions are generate using a Monte Carlo Glauber model based on a common two-component ansatz which deposits entropy proportional to a linear combination 
of nucleon participants and binary nucleon-nucleon collisions,
\begin{equation}
 dS/dy \,\vert_{y=0} \propto \frac{(1-\alpha)}{2}N_\text{part} + \alpha N_\text{coll}
 \label{twocomponent}
\end{equation}
where for the binary collision fraction, we use $\alpha=0.14$ which has been shown to provide a good description of the centrality dependence of charged particle 
multiplicity in $200$ GeV gold-gold collisions \cite{?}.

The entropy is localized about each nucleon's transverse parton density $T_p({\bf x})$,
\begin{eqnarray}
 dS/dy \,\vert_{y=0} &\propto& \sum\limits_{i=0}^{N_\text{part,A}} w_i\, T_p({\bf x} - {\bf x}_i)(1-\alpha + \alpha\, N_\text{coll,i}) \nonumber \\
                     &+& \sum\limits_{j=0}^{N_\text{part,B}} w_j\, T_p({\bf x} - {\bf x}_i)(1-\alpha + \alpha\, N_\text{coll,j}),
 \label{glauber}
\end{eqnarray}
where the summations run over the participants in each nucleus, $N_\text{coll,i}$ denotes the number of binary collisions suffered by the $i^\text{th}$ nucleon 
and the proton density $T_p({\bf x})$ is described by a Gaussian
\begin{equation}
 T_p({\bf x}) = \frac{1}{\sqrt{2 \pi B}} \exp \left(-\frac{x^2+y^2}{2 B} \right)
\end{equation}
with transverse area $B = 0.36$ $\text{fm}^2$.

The random nucleon weights $w_i$ in equation \eqref{glauber} are sampled independently from a Gamma distribution with unit mean
\begin{equation}
 P_k(w) = \frac{k^k}{\Gamma(k)} w^{k-1} e^{-k w},
\end{equation}
and shape parameter $k = \text{Var}(P)^{-1}$ which modulates the variance of the distribution. 
These fluctuations are typically added \cite{?} to reproduce the large multiplicity fluctuations observed in minimum bias proton-proton collisions. 
In this work the shape parameter is fixed to $k=1$ determined by a fit to the $200$ GeV UA5 data \cite{?}. 

The initial condition profiles, which provide the entropy density $dS/(d^2r_\perp\, d\eta\, \tau_\text{therm})$ at the QGP thermalization time, are finally 
rescaled by an overall normalization factor to fit the measured charged particle multiplicity in $0$--$10\%$ centrality collisions.

\begin{figure}
  \includegraphics[width=\columnwidth]{./fig/cs}
  \caption{\label{fig:cs} Speed of sound squared $c_s^2$ plotted versus temperature $T$ for the three equations of state used in this study. The vertical
	   gray line indicates the switching temperature $T_\text{sw} = 154$ MeV where the model switches from fluid dynamics to a microscopic transport model.}
\end{figure}

\begin{figure*}[t]
  \includegraphics[width=\textwidth]{./fig/spectra}
  \caption{
    \label{fig:spectra} Effect of the equation of state on transverse momentum spectra. Top row: model calculations using the HotQCD equation of state plotted against 
    PHENIX data for pions, kaons and protons (blue lines/circles, red lines/squares and green lines/triangles) in centrality bins $0$--$5\%$, $20$--$30\%$ and $40$--$50\%$ 
    (columns left to right). Middle and bottom rows: ratios of the WB' and S95' invariant yields to the HotQCD' result. Shaded bands indicate two sigma statistical error. }
\end{figure*}

\begin{figure*}[t]
  \includegraphics[width=\textwidth]{./fig/v2}
  \caption{
    \label{fig:v2} Effect of the equation of state on differential elliptic flow $v_2(p_T)$ calculated from the Cooper-Frye freezeout hypersurface \eqref{differential_flow}.
    Top row: model calculations using the HotQCD' equation of state for the elliptic flow $v_2(p_T)$  of pions, kaons and protons (blue, orange and green lines) 
    in centrality bins $0$--$10\%$, $20$--$30\%$ and $40$--$50\%$ (columns left to right). Middle and bottom rows: ratios of the WB' and S95' elliptic flow to 
    the HotQCD' result. Statistical errors are smaller than the linewidth and have been omitted.
  }
\end{figure*}

\begin{figure*}[t]
  \includegraphics[width=\textwidth]{./fig/v3}
  \caption{
    \label{fig:v3} Same as Fig.~\ref{fig:v2} but for differential triangular flow $v_3(p_T)$. Note that the y-axis limits in the top row are different.
  }
\end{figure*}

\begin{figure*}[t]
  \includegraphics[width=\textwidth]{./fig/hbt}
  \caption{
    \label{fig:hbt} Effect of the equation of state on the Bertsch-Pratt radii.  We plot $R_o$, $R_s$, $R_l$ and the ratio $R_o/R_s$ (rows top to bottom) 
    in centrality bins $0$--$10\%$, $20$--$30\%$ and $40$--$50\%$ (columns left to right) against transverse mass $m_T$ for the HotQCD', WB'  and S95' equations of state 
    (purple, orange and blue lines). Shaded bands indicate two sigma errors estimated from the Jacobian of the fit function \eqref{fitfunction}. Symbols with errors bars 
    are experimental data from PHENIX.  
  }
\end{figure*}

\subsection{Hydrodyamics and Boltzmann transport}

The hydrodynamic equations of motion are obtained in VISHNew by solving the second-order Israel-Stewart equations,
\begin{equation}
 \partial_\mu T^{\mu\nu} = 0, \quad T^{\mu\nu} = e u^\mu u^\nu - (p + \Pi) \Delta^{\mu\nu} + \pi^{\mu\nu},
\end{equation}
where the bulk pressure $\Pi$ and shear stress $\pi^{\mu\nu}$ satisfy the relaxation equations,
\begin{eqnarray}
 \label{viscosity}
 \cr \mathcal{D}\Pi = &-&\frac{1}{\tau_\Pi}(\Pi + \zeta \theta) - \frac{1}{2} \Pi \frac{\zeta T}{\tau_\Pi}d_\lambda \left(\frac{\tau_\Pi}{\zeta T} u^\lambda \right), \nonumber \\
  \Delta^{\mu\alpha} \Delta^{\nu\beta} \mathcal{D}\pi_{\alpha\beta} = &-&\frac{1}{\tau_\pi}(\pi^{\mu\nu} - 2 \eta \sigma^{\mu\nu}) \\
  &-& \frac{1}{2} \pi^{\mu\nu} \frac{\eta T}{\tau_\pi} d_\lambda\left(\frac{\tau_\pi}{\eta T} u^\lambda \right ).
\end{eqnarray}

We follow the work in reference \cite{?} and fix the bulk viscosity $\zeta$ and shear viscosity $\eta$ in equation \eqref{viscosity} using a constant specific shear viscosity $\eta/s=0.08$ 
and vanishing bulk viscosity $\zeta/s=0$ in the hydrodynamic phase of the simulation. It would be interesting to study the effect of bulk viscous corrections which are sensitive to 
the peak of the QCD trace anomaly near the QGP phase transition \cite{?}. Unfortunately, bulk viscous corrections do not have a straight forward implementation in the present hybrid model 
and are neglected in this work. 

As previously explained in section \,\ref{eos}, the VISHNU hybrid model transitions from hydrodyamic field equations to microscopic transport at a sudden switching temperature $T_\text{sw}$ 
at which the hydrodynamic energy-momentum tensor is particlized using the Cooper-Frye freezeout prescription,
\begin{equation}
 E\frac{dN_i}{d^3p} = \int_\sigma f_i(x,p) p^\mu d^3\sigma_\mu
 \label{cooper-frye}
\end{equation}
where $f_i$ is the distribution function of particle species $i$, $p^\mu$ is its four-momentum and $d^3\sigma_\mu$ characterizes an element of the isothermal freezeout 
hypersurface defined by $T_\text{sw}$.

The sampled particles then enter the UrQMD simulation where the Boltzmann equation, 
\begin{equation}
 \frac{df_i(x,p)}{dt} = \mathcal{C}_i(x,p),
\end{equation}
is solved to simulate all elastic and inelastic collisions between the particles with collision kernel $\mathcal{C}_i$ until the system becomes too dillute to continue interacting. 
Finally, the four-position, four-momentum and particle identification number of each particle recorded at the moment of last interaction. 
 
\section{Results}

The results section is organized as follows. In sub-section \ref{spectra} we calculate the particle spectra for each equation of state across three different centrality classes using the final 
particle information output by the hybrid simulation. In sub-section \ref{flow} we repeat the calculation for elliptic and triangular flow but perform the calculation on the 
hydrodynamic Cooper-Frye freezeout surface for reasons explained later in the text. In sub-section \ref{hbt} we calculate the femptoscopic event-averaged Bertsch-Pratt radii, again using the 
final particle information output by the full hybrid calculation. Finally in sub-section \ref{errors}, we repeat the spectra and flow analysis using a sampling of equation of state curves from the 
HotQCD published errors. 

All results presented in the following sections are based on $10^5$ minimum bias events which are subdivided into centrality classes according to initial entropy, e.g. the initial condition events 
with $20\%$ highest entropy comprise centrality class $0$--$20\%$. Each hydrodynamic event is oversampled an additional four times to increase the number of particles in each event and suppress 
finite statistical error.

\subsection{Particle spectra}
\label{spectra}

Figure \ref{fig:spectra} shows the invariant yield $dN/(2\pi p_T dp_T dy)$ of positively charged pions, kaons and protons calculated from the hybid model for the $0$--$5\%$, $20$--$30\%$
and $40$--$50\%$ centrality classes using the HotQCD', WB' and S95' equations of state constructed in section \ref{eos}. 

The first row shows the HotQCD' yields obtained from the hybrid model plotted against observed pion, proton and kaon data from PHENIX. The second and third rows show the ratio of the invariant yields of 
the WB' and S95' equations of state over the the HotQCD' result. We see that the HotQCD' equation of state provides a good description of observed particle yields except for at moderate 
to large $p_T$ in central collisions where the equation of state overpredicts the data. This agreement would likely improve with more realistic initial conditions, bulk viscous corrections 
and/or more careful treatment of the hydro-to-micro switching temperature $T_\text{sw}$, and thus we defer from making any specific conclusions from the overall fit to data. It suffices to say that the 
most recent HotQCD lattice results provide a reasonable description of the PHENIX data and agrees within the overall uncertainty of the present model. 

Looking at the second and third rows of the figure which show the ratios of the WB' and S95' yields to the HotQCD' result, we see that the spectra predicted by the HotQCD' and WB' equations of state 
agree within statistical error, while the S95' equation of state is appreciably softer and produces $\sim 10\%$ more particles at $p_T = 0.5$ GeV and $\sim 30\%$ fewer particles at $p_T=2.5$ GeV across all three centralities.

\subsection{Elliptic and triangular flows}
\label{flow}

The azimuthal anisotropy of final particle emission is characterized by the Fourier expansion
\begin{equation}
 E \frac{d^3N}{d^3p} = \frac{1}{2\pi} \frac{d^2N}{dy p_T dp_T} \left(1 + \sum\limits_{n=1}^\infty 2 v_n \cos n(\phi - \Psi_{RP}) \right)
\end{equation}
where $\phi$ is the direction of the emitted particle, $\Psi_{RP}$ is the reaction plane angle of the event and $v_n$ the anisotropic flow coefficient corresponding to the Fourier harmonic of order $n$.

The reaction plane angle cannot be measured experimentally and the anisotropic flow is typically estimated using multi-particle correlations such as two and four-particle cumulants. The statistical error of the 
event-averaged estimators is suppressed with both increasing event multiplicity and event sample size. This can pose a challenge for computationally intensive hybrid model calculations which typically cannot 
reach integrated luminosities comparable to experiment. 

Statistical errors are particularly noxious in differential flow calculations at moderate to large $p_T$ where particle statistics are limited. We circumvent this issue in the differential flow 
analysis and calculate the flow anisotropy of pions, kaons and protons directly from the Cooper-Frye freezeout surface using the built in routines in the VISHNU package according to
\begin{equation}
 \label{differential_flow}
 v_n(p_T) = \frac{\int d\phi_p e^{i n \phi_p} dN/(dy p_T dp_T d\phi_p)}{\int d\phi_p\, dN/(dy p_T dp_T d\phi_p)}.
\end{equation}
Consequently, the present flow sensitivity analysis does not include contributions from flow generated by the UrQMD hadronic afterburner which is indentical for each of the three equations of state. Hence, the following
results should be interpretted as a conservative \emph{upper} bound on the goodness of fit sensitivity expected in a full hybrid model simulation.

Figs.~\ref{fig:v2} shows the elliptic flow $v_2$ of pions, kaons and protons calculated from equation \eqref{differential_flow} for the HotQCD', WB' and S95' equations of state in $0$--$10$, $20$--$30$ and $40$--$50\%$ centrality bins. 
The first row of the figure shows the elliptic flow predicted by the HotQCD' equation of state while the middle and bottom rows display theoretical ratios of the WB' and S95' predictions over the HotQCD' result. The information
in Fig.~\ref{fig:v3} is identical to that in Fig.~\ref{fig:v2} except that elliptic flow $v_2$ has been replaced with triangular flow $v_3$.

We see in Fig.~\ref{fig:v2} that the elliptic flow generated by the HotQCD' and WB' parameterizations is in very good agreement across all centralities, while the S95' parameterization systematically generates $\sim 10\%$ less 
flow than the HotQCD' equation of state. This is consistent with previous findings that the S95' equation of state is considerably softer in the vicinity of the phase transition as evidenced by the 
speed of sound in Fig.~\ref{fig:cs}. 

In Fig.~\ref{fig:v3}, we see that the effect on the triangular flow is similar to the effect observed on the elliptic flow except more pronounced and generates as large as a $20\%$ discrepancy in the peripheral flows
predicted by the HotQCD' and S95' equations of state. This sensitivity of higher harmonics to the softness of the QGP phase equation of state puts the relatively large higher-order anisotropic flow coefficients observed at
RHIC and the LHC into perspective. 

\subsection{Femptoscopic Bertsch-Pratt radii}
\label{hbt}

The size of the fireball emission region is estimated using Hanbury-Brown-Twiss (HBT) interferometry for identical particles. The azimuthally averaged two-particle correlation function 
\begin{equation}
 \label{hbt}
 C(q, k) = \frac{\sum\limits_n \sum\limits_{i, j} \delta_{q} \, \delta_{k}\Psi(q,r)}{\sum\limits_{n} \sum\limits_{i,j'} \delta_{q} \, \delta_{k}}
\end{equation}
consists of a numerator with particles pairs sampled from the same event and a denominator with pairs sampled from different events. Here $q = p_i - p_j$ denotes the relative momentum, $r=x_i-x_j$ the relative separation and $k = (p_i + p_j)/2$ the average momentum of the pion pair in the longitudinal co-moving 
frame where the compononent of $k$ along the beam axis vanishes. The numerator is summed over all events $n$ in a given centrality class and unique particle pair combinations $i,j$ in each event. In the denominator, particle $i$ is
taken from one event and particle $j'$ from a random partner event in the same centrality class. The delta functions $\delta_q$ and $\delta_k$ are $1$ if the momenta $q$ and $k$ fall into their respective bins and $0$ otherwise. Bose-Einsten 
correlations, which are not included natively in the UrQMD model, are imposed by adding the symmetrization factor $\Psi(q,r) = 1 + \cos q\,r$. 

The average pair momentum $k$ is then projected into its longitudinal component $k_\text{z}$ and transverse component $k_T$, while the separation momentum $q$ is represented in the orthogonal coordinates 
$(q_o, q_s, q_l)$, where $q_l$ lies along the beam axis, $q_o$ is parallel to $k_T$ and $q_s$ perpendicular to $q_o$ and $q_l$. The resulting correlation function is approximated using a Gaussian source and 
fit to the parametric form
\begin{equation}
 \label{fitfunction}
 C(q_o, q_s, q_l, k_T) = \mathcal{N} \left(1 + \lambda\, e^{-R_o^2 q_o^2 - R_s^2 q_s^2 - R_l^2 q_l^2} \right) 
\end{equation}
by finding the best fit normalization $\mathcal{N}$, source strength $\lambda$ and Bertsch-Pratt radii $R_o$, $R_s$ and $R_l$ for a given transverse momentum $k_T$. 

We calculate the Bertsch-Pratt radii for each equation of state using identical pions. The fit is perfomed using $2\cdot10^4$ hydrodynamic events in each centrality bin and an additional $10$ UrQMD oversamples per event. 
The oversamples are then concatenated into a single particle list to increase the number of particle pairs by a factor $10^2$.

In Fig.~\ref{fig:hbt}, we plot the Bertsch-Pratt radii for the HotQCD', WB' and S95' equations of state as functions of the transverse mass $m_T = \sqrt{m^2 + k_T^2}$ . The horizontal rows show the radii $R_o$, $R_s$, $R_l$ and ratio $R_o/R_s$ (top to bottom), while the columns 
mark centrality classes $0$--$10\%$, $10$--$20\%$ and $20$--$40\%$ (left to right). The different colored lines annotated in the legend indicate different equations of state and the bands estimate the error from the Jacobian 
of the fit. The symbols with error bars are experimental data from PHENIX.

We see that the Bertsch-Pratt radii predicted by the hybrid model provide a good description of the data across all centralities, except for at low $p_T$ where $R_o$ and $R_l$ slightly undershoot the data. However, in contrast to
spectra and flows we see no discernible difference in the Bertsch-Pratt radii predicted by the three different equations of state. This suggests that femptoscopic measurements are not sensitive enough to resolve small differences
in the lattice equation of state.
 
\subsection{HotQCD errors}
\label{errors}

As a final test we analyze errors in the 

\section{Conclusion and Outlook}

\begin{acknowledgments}
 JSM acknowledges support by the DOE/NNSA Stockpile Stewardship Graduate Fellowship under grant no.~DE-FC52-08NA28752.
\end{acknowledgments}

\bibliography{eos,duke-qcd-refs/Duke_QCD_refs}


\end{document}
