\documentclass[aps,prc,reprint,amsmath,nofootinbib,superscriptaddress]{revtex4-1}

\usepackage{hyperref}
\usepackage{graphicx}
\usepackage{amsmath}
\graphicspath{{fig/}}

\usepackage{mdwlist}

\newcommand{\nch}{N_\text{ch}}

\begin{document}

\title{Hydrodynamic simulations of relativistic heavy-ion collisions\\ with different calculations of the QCD equation of state}

\author{J.\ Scott Moreland}
\affiliation{Department of Physics, Duke University, Durham, NC 27708-0305}
\author{Ron A.\ Soltz}
\affiliation{Lawrence Livermore National Laboratory, Livermore, CA 94551-0808}

\date{\today}

\begin{abstract} 
   
\end{abstract}

\maketitle

\section{Introduction}

Fluid dynamics is a useful framework to study the collective behaviour of hot and dense nuclear matter produced in relativistic heavy-ions. 
Quantum Chromodynamics (QCD) predicts that at sufficiently high energies these collisions form a new state of matter consisting of deconfined 
quarks and gluons known as a quark-gluon plasma (QGP). Simulations based on relativistic viscous hydrodynamics play a central role in extracting 
properties of the QGP which expands and freezes into hadrons too quickly for direct observation.

The hydroynamic transport equations require two essential ingredients to specify the full time evolution of the QGP fireball: initial conditions 
which describe the thermal profile of the QGP droplet at some early starting time and a QCD equation of state which interrelates energy density, 
pressure and temperature of each fluid cell in local thermal equillibrium.

Lattice discretization is the only reliable method to calculate the QCD equation of state in the vicinity of the QGP phase transition and hence 
constitutes a critical component of hydrodynamic simulations. While lattice techniques are rigorous in their treatment of the underlying QCD 
Lagrangian, they are subject to numerical errors inherent in the lattice discretization procedure. These errors are manifest in differences in the 
continuum extrapolated QCD trace anomaly predicted by different lattice collaborations and lead to an overall uncertainty in the true value of the 
QCD equation of state at zero baryochemical potential.

Simulations using a lattice based equation of state inherrit all forms of numeric and systematic uncertainty associated with the underlying 
lattice methodology. These modeling uncertainties have been studied both at low temperature, by comparing simulations with a lattice equation of 
state to results obtained from a hadron resonance gas model \cite{Huovinen:2009yb}, and at high temperature by comparing the effect of different 
lattice parameterizations of the lattice equation of state on particle spectra and flow \cite{Huovinen:2005gy, Huovinen:2009yb}. 

Recent calculations by the HotQCD and Wuppertal-Budapest collaborations of the QCD trace anomaly in the continuum limit now show good agreement within 
errors. This signals an important convergence in lattice descriptions of the QCD equation of state which previously exhibited a tension in the peak 
of the trace anomaly near the QGP phase transition. It is not yet clear however, if current lattice errors are under sufficient control for 
hydrodynamic transport models or if further improvement is needed.  

In this work, we quantify the effect of lattice errors on simulations of relativistic heavy-ion collisions by comparing simulation predictions obtained 
with different calculations of the QCD equation of state. The equations of state are analyzed using a modern event-by-event hybrid simulation which couples
viscous hydrodynamics to a hadronic afterburner to calculate flows, spectra and Bertsch-Pratt radii and are compared to measurements at the Relativistic
Heavy-Ion Collider (RHIC).
 
We analyze two state of the art calculations from the HotQCD and Wuppertal-Budapest collaborations as well as an older parameterization based on HotQCD 
calculations with a coarser lattice spacing to gauge the importance of recent lattice improvements. We also assess the need for additional improvements 
to the current state of the art by sampling equation of state curves from the latest HotQCD published errors to quantify current uncertainties. 



\section{Equations of State}
\label{eos}

The hybrid simulation used in this work switches from viscous hydrodynamics to a microscopic kinetic description once the system expands, cools and freezes 
into hadrons. While the QCD equation of state enters the hydrodynamic phase of the simulation as a freely specified function interrelating energy density, 
pressure and temperature, its description in the kinetic phase of the collision is fixed by the finite number of particles and particle resonances included 
in the UrQMD model used for the hadronic phase of the simulation.

As a result, we limit our study to differences in the QCD equation of state \emph{above} the QGP transition temperature where hydrodynamics allows us to 
freely vary its chosen form. We study three different parameterizations for this high temperature dependence -- two state of the art calculations 
in 2+1 flavour QCD from the HotQCD \cite{Bazavov:2014pvz} and Wuppertal-Budapest \cite{Borsanyi:2013bia} collaborations, as well as the older s95p-v1 
parameterization \cite{Huovinen:2009yb} constructed using lattice data measured with a coarser lattice spacing \cite{Bazavov:2009zn}.

The QCD equation of state is frequently characterized by the trace of the energy-momentum tensor, also referred to as the trace anomaly or interaction measure. 
When scaled by powers of the temperature, the trace anomaly forms a dimensionless measure
\begin{equation}
 I \equiv \frac{\Theta^{\mu\mu}(T)}{T^4} = \frac{e - 3p}{T^4},
\end{equation}
where $e$ is the local fluid energy density, $p$ the pressure and $T$ the temperature.

In Fig.~\ref{fig:trace} we plot the temperature scaled interaction measure of each equation of state as well as that of a hadron resonance gas calculated from 
the list of partial resonances included in the UrQMD collision kernel. The s95p-v1 parameterization, which was constructed to interpolate between 
a hadron resonance gas calculation at low temperatures and lattice results at high temperatures, is in good agreement with the UrQMD equation of state while the 
HotQCD and Wuppertal-Budapest results are slightly higher in the vicinity of the phase transition.

To ensure a self consistent description in regions of the collision where the simulation switches from hydrodynamics to Boltzmann transport, we match each high 
temperature lattice equation of state with the low temperature UrQMD equation of state. We thus define a piecewise function for the temperature scaled interaction measure,
\begin{equation}
 \label{interaction}
 I(T) =
  \begin{cases}
   I_\text{hrg}(T)	& T \le T_1 \\
   I_\text{blend}(T)	& T_1 < T < T_2 \\ 
   I_\text{lattice}(T)	& T \ge T_2,
  \end{cases}
\end{equation}
where $I_\text{hrg}$ is the hadron resonance gas trace anomaly in UrQMD pictured in Fig.~\ref{fig:trace}, $I_\text{lattice}$ represents one of the HotQCD, Wuppertal-Budapest or S95p-v1 
parameterizations and $I_\text{blend}$ is a function which smoothly connects between the two in the temperature interval $T_1 < T < T_2$,
\begin{equation}
  \label{interpolation}
  I_\text{blend} = (1-z)\, I_\text{hrg} + z\, I_\text{lattice}.
\end{equation}
The interpolation parameter $z \in [0,1]$ in equation \ref{interpolation} is constructed to match the first and second derviatives at the endpoints of the 
interpolation interval,
\begin{eqnarray}
 \label{smoothstep}
 \cr z &=& 6 x^5 - 15 x^4 + 10 x^3 \\
  \text{where } x &=& (T - T_1)/(T_2 - T_1),
\end{eqnarray}
where the endpoints $T_1 = T_\text{sw}$ and $T_2 = 180$ MeV smoothly interpolate the lattice results into the UrQMD trace anomaly at the desired switching 
temperature $T_\text{sw}$.

\begin{figure}[t]
  \includegraphics[width=\columnwidth]{./fig/trace}
  \caption{\label{fig:trace} The temperature scaled QCD trace anomaly for the UrQMD. HotQCD, WB and s95p-v1 parameterizations as a function of temperature \cite{?}.}
\end{figure}

\begin{figure}[b]
  \includegraphics[width=\columnwidth]{./fig/trace_final}
  \caption{\label{fig:trace_final} The modified QCD trace anomalies HotQCD', WB' and S95' obtained from equation \eqref{interaction} and the corresponding lattice
	  parameterizations in Fig.~\ref{fig:trace}. The gray, vertical line marks the hydro-to-micro switching temperature $T_\text{sw} = 154$ MeV.}
\end{figure}

In principle, the switching temperature could assume any value in a small interval below the equation of state's pseudo-critical transition temperature $T_\text{c}$ 
and should be tuned to fit the relative abundance of pions, protons and kaons measured by experiment. Since we are primarily interested in the sensitivity of the 
simulation to changes in the equation of state with all other quantities held fixed, we fix the transition temperature using the HotQCD chiral transition temperature 
$T_\text{sw} = T_\text{c} = 154$ MeV.
 
The modified interaction measures, labeled with a prime to distinguish them from the raw lattice results, are plotted in Fig.~\ref{fig:trace_final}. The vertical gray line 
marks the hydro-to-micro switching temperature $T_\text{sw}=154$ MeV where the model switches from the VISH2+1 hydrodynamics code to UrQMD. 

In Fig.~\ref{fig:cs} we plot the squared speed of sound $c_s^2 = dp / de$ for each modified interaction measure. The speed of sound of the HotQCD' and WB' equations
of state are in good agreement while the S95' parameterization remains softer in a wider interval about the QGP phase transition. We note that the speed of sound in 
the HotQCD' and WB' parameterizations is clearly affected by the parametric transition \eqref{interpolation} in the vicinity of the switching temperature (vertical gray line), 
but that the imposed matching maintains continuity.

With the trace anomalies in hand, the energy density, pressure and entropy density are easily interrelated to specify the equation of state used in the analysis,
\begin{eqnarray}
 \cr \frac{p(T)}{T^4} &=& \int\limits_0^T dT'\, \frac{I(T')}{T'}, \\
 \frac{e(T)}{T^4} &=& I(T) + 3\, \frac{p(T)}{T^4}, \\
 \frac{s(T)}{T^3} &=& \frac{e(T) + p(T)}{T^4}. 
\end{eqnarray}
 

\section{Hybrid Model}

The equations of state are embedded in the modern event-by-event VISHNU hybrid model which uses the VISH2+1 boost-invariant viscous hydrodynamics code to simulate the 
time evolution of the QGP medium and a microscopic UrQMD hadronic afterburner for subsequent evolution below the QGP transition temperature. Where necessary, free
parameters of the model are are tuned to facillitate model-to-data comparison with $200$ GeV gold-gold collisions at RHIC. In this section, we briefly outline
the implementation of the model used in the analysis; for a more detailed explanation of the VISHNU model, we refer the reader to \cite{}. 

\subsection{Initial Conditions}

The hydrodynamic initial conditions are generate using a Monte Carlo Glauber model based on a common two-component ansatz which deposits entropy proportional to a linear combination 
of nucleon participants and binary nucleon-nucleon collisions,
\begin{equation}
 dS/dy \,\vert_{y=0} \propto \frac{(1-\alpha)}{2}N_\text{part} + \alpha N_\text{coll}
 \label{twocomponent}
\end{equation}
where for the binary collision fraction, we use $\alpha=0.14$ which has been shown to provide a good description of the centrality dependence of charged particle 
multiplicity in $200$ GeV gold-gold collisions \cite{?}.

The entropy is localized about each nucleon's transverse parton density $T_p({\bf x})$,
\begin{eqnarray}
 dS/dy \,\vert_{y=0} &\propto& \sum\limits_{i=0}^{N_\text{part,A}} w_i\, T_p({\bf x} - {\bf x}_i)(1-\alpha + \alpha\, N_\text{coll,i}) \nonumber \\
                     &+& \sum\limits_{j=0}^{N_\text{part,B}} w_j\, T_p({\bf x} - {\bf x}_i)(1-\alpha + \alpha\, N_\text{coll,j}),
 \label{glauber}
\end{eqnarray}
where the summations run over the participants in each nucleus, $N_\text{coll,i}$ denotes the number of binary collisions suffered by the $i^\text{th}$ nucleon 
and the proton density $T_p({\bf x})$ is described by a Gaussian
\begin{equation}
 T_p({\bf x}) = \frac{1}{\sqrt{2 \pi B}} \exp \left(-\frac{x^2+y^2}{2 B} \right)
\end{equation}
with transverse area $B = 0.36$ $\text{fm}^2$.

The random nucleon weights $w_i$ in equation \eqref{glauber} are sampled independently from a Gamma distribution with unit mean
\begin{equation}
 P_k(w) = \frac{k^k}{\Gamma(k)} w^{k-1} e^{-k w},
\end{equation}
and shape parameter $k = \text{Var}(P)^{-1}$ which modulates the variance of the distribution. 
These fluctuations are typically added \cite{?} to reproduce the large multiplicity fluctuations observed in minimum bias proton-proton collisions. 
In this work the shape parameter is fixed to $k=1$ determined by a fit to the $200$ GeV UA5 data \cite{?}. 

The initial condition profiles, which provide the entropy density $dS/(d^2r_\perp\, d\eta\, \tau_\text{therm})$ at the QGP thermalization time, are finally 
rescaled by an overall normalization factor to fit the measured charged particle multiplicity in $0$--$10\%$ centrality collisions.

\begin{figure}
  \includegraphics[width=\columnwidth]{./fig/cs}
  \caption{\label{fig:cs} Speed of sound squared $c_s^2$ plotted versus temperature $T$ for the three equations of state used in this study. The vertical
	   gray line indicates the switching temperature $T_\text{sw} = 154$ MeV where the model switches from fluid dynamics to a microscopic transport model.}
\end{figure}

\begin{figure*}[t]
  \includegraphics[width=\textwidth]{./fig/spectra}
  \caption{
    \label{fig:spectra} Invariant yields of the HotQCD', WB' and S95' equations of state for positively charged pions (blue/circles), kaons (red/squares) and 
    protons (green/triangles) in centrality bins $0$--$5$\% (left column), $20$--$30$\% (middle column) and $40$--$50$\% (right column). The top row shows the 
    HotQCD' simulation result (lines) plotted against data from PHENIX (symbols). The middle and bottom rows show the ratio of the WB' and S95' invariant yields 
    to the HotQCD' result (lines with bands). Pions and kaons have been offset for clarity.
  }
\end{figure*}

\begin{figure*}[t]
  \includegraphics[width=\textwidth]{./fig/v2}
  \caption{
    \label{fig:spectra}
  }
\end{figure*}

\begin{figure*}[t]
  \includegraphics[width=\textwidth]{./fig/v3}
  \caption{
    \label{fig:spectra}
  }
\end{figure*}

\begin{figure*}[t]
  \includegraphics[width=\textwidth]{./fig/hbt}
  \caption{
    \label{fig:spectra}
  }
\end{figure*}

\subsection{Hydrodyamics and Boltzmann Transport}

The hydrodynamic equations of motion are obtained in VISHNew by solving the second-order Israel-Stewart equations,
\begin{equation}
 \partial_\mu T^{\mu\nu} = 0, \quad T^{\mu\nu} = e u^\mu u^\nu - (p + \Pi) \Delta^{\mu\nu} + \pi^{\mu\nu},
\end{equation}
where the bulk pressure $\Pi$ and shear stress $\pi^{\mu\nu}$ satisfy the relaxation equations,
\begin{eqnarray}
 \label{viscosity}
 \cr \mathcal{D}\Pi = &-&\frac{1}{\tau_\Pi}(\Pi + \zeta \theta) - \frac{1}{2} \Pi \frac{\zeta T}{\tau_\Pi}d_\lambda \left(\frac{\tau_\Pi}{\zeta T} u^\lambda \right), \nonumber \\
  \Delta^{\mu\alpha} \Delta^{\nu\beta} \mathcal{D}\pi_{\alpha\beta} = &-&\frac{1}{\tau_\pi}(\pi^{\mu\nu} - 2 \eta \sigma^{\mu\nu}) \\
  &-& \frac{1}{2} \pi^{\mu\nu} \frac{\eta T}{\tau_\pi} d_\lambda\left(\frac{\tau_\pi}{\eta T} u^\lambda \right ).
\end{eqnarray}

We follow the work in reference \cite{?} and fix the bulk viscosity $\zeta$ and shear viscosity $\eta$ in equation \eqref{viscosity} using a constant specific shear viscosity $\eta/s=0.08$ 
and vanishing bulk viscosity $\zeta/s=0$ in the hydrodynamic phase of the simulation. It would be interesting to study the effect of bulk viscous corrections which are sensitive to 
the peak of the QCD trace anomaly near the QGP phase transition \cite{?}. Unfortunately, bulk viscous corrections do not have a straight forward implementation in the present hybrid model 
and are neglected in this work. 

As previously explained in section \,\ref{eos}, the VISHNU hybrid model transitions from hydrodyamic field equations to microscopic transport at a sudden switching temperature $T_\text{sw}$ 
at which the hydrodynamic energy-momentum tensor is particlized using the Cooper-Frye freezeout prescription,
\begin{equation}
 E\frac{dN_i}{d^3p} = \int_\sigma f_i(x,p) p^\mu d^3\sigma_\mu
 \label{cooper-frye}
\end{equation}
where $f_i$ is the distribution function of particle species $i$, $p^\mu$ is its four-momentum and $d^3\sigma_\mu$ characterizes an element of the isothermal freezeout 
hypersurface defined by $T_\text{sw}$.

The sampled particles then enter the UrQMD simulation where the Boltzmann equation, 
\begin{equation}
 \frac{df_i(x,p)}{dt} = \mathcal{C}_i(x,p),
\end{equation}
is solved to simulate all elastic and inelastic collisions between the particles with collision kernel $\mathcal{C}_i$ until the system becomes too dillute to continue interacting. 
Finally, the four-position, four-momentum and particle identification number of each particle recorded at the moment of last interaction. 
 
\section{Analysis and Results}

The results are presented in two


\section{Results}


\section{Conclusion}

\section{Summary}

\begin{acknowledgments}
 JSM acknowledges support by the DOE/NNSA Stockpile Stewardship Graduate Fellowship under grant no.~DE-FC52-08NA28752.
\end{acknowledgments}

\bibliography{eos,duke-qcd-refs/Duke_QCD_refs}


\end{document}
