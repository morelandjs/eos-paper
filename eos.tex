\documentclass[aps,prc,reprint,amsmath,nofootinbib,superscriptaddress]{revtex4-1}

\usepackage{hyperref}
\usepackage{graphicx}
\graphicspath{{fig/}}

\usepackage{mdwlist}

\newcommand{\nch}{N_\text{ch}}

\begin{document}

\title{Quantifying uncertainty in hydrodyamic simulations of heavy-ion collisions attributable to imperfect knowledge of the 
QCD equation of state}

\author{J.\ Scott Moreland}
\affiliation{Department of Physics, Duke University, Durham, NC 27708-0305}
\author{Ron A.\ Soltz}
\affiliation{Lawrence Livermore National Laboratory, Livermore, CA 94551-0808}

\date{\today}

\begin{abstract} 
   
\end{abstract}

\maketitle

\section{Introduction}

Hydrodynamic simulations are a popular tool to model the spacetime evolution of the quark-gluon plasma (QGP) produced in relativistic 
heavy-ion collisions. Lattice regularization is the only reliable method to calculate the QCD equation of state equation of state in 
the vicinity of a phase transition and hence constitutes a critical component of modern computer simulations. While lattice techniques 
are rigorous in their treatment of the underlying QCD Lagrangian, they are subject to numerical errors inherent in the lattice discretization 
procedure. These errors are manifest in differences in the continuum extrapolated QCD trace anomaly predicted by different 
lattice collaborations and lead to an overall uncertainty in the true value of the QCD equation of state at zero baryochemical potential.

For the purposes of hydrodynamic simulations, the equation of state is typically treated as a theoretically constrained quantity in contrast 
to e.g. the QGP specific shear viscosity $\eta/s$ which is varied and tuned to optimally replicate experimental data. Consequently, numerical 
discrepancies between different lattice collaborations introduce an inherent systematic bias in the best fit values of underconstrained QGP 
properties determined from systematic model-to-data comparison. A notable exception to this convention is a recent model-to-data analysis 
which parameterized the QGP equation of state and used a Bayesian, data driven approach to constrain its functional form \cite{Novak:2013bqa}. 

Uncertainties in the equation of state have been studied both at low temperature, by comparing lattice predictions to results from a hadron 
resonance gas model \cite{Huovinen:2009yb}, and at high temperature by comparing hydrodynamic predictions obtained using different 
parameterizations of the QCD trace anomaly \cite{Huovinen:2005gy, Huovinen:2009yb}. Large differences were 

Recent calculations by the HotQCD and Wuppertal-Budapest collaborations of the QCD trace anomaly in the continuum limit now show good agreement 
within errors. This signals an important convergence in lattice descriptions of the QCD equation of state which previously exhibited a tension 
in the peak of the trace anomaly near the QGP phase transition. It is not yet clear however, if current lattice errors are under sufficient 
control for hydrodynamic transport models or if further improvement is needed.  

In this work, we analyze the current status of lattice gauge calculations in the continuum limit by comparing simulation predictions using 
different lattice calculations for the QGP equation of state. We study the latest HotQCD and Wuppertal-Budapest parametrizations as well as 
the depreciated s95 parametrization based on older HotQCD lattice results. 

We embed each equation of state in an event-by-event hydrodynamic model with a hadronic afterburner and measure spectra, flows and Bertsch-Pratt 
radii predicted by the simulations to quantify systematic differences between the calculations. We also assess the uncertainty introduced by 
continuum extrapolation when constructing the best fit parametrization by sampling splines from the bootstrap coefficients used in the HotQCD 
error analysis. Using these results, we comment on the resolving power of hydrodynamic simulations and assess the need for improved lattice 
calculations at zero baryo-chemical potential.

\section{Hybrid Model}

The equations of state are compared using the VISHNU transport model which couples boost invariant viscous fluid dynamics \cite{?} for the hot 
and dense early phase of the collision with a microscopic, kinetic description of late hadronic rescattering and freeze-out \cite{?}. The mock 
particle data generated by each simulated event are then stored and analyzed using the same methods applied in the experiment.

\subsection{Initial Conditions}

\begin{figure}
  \includegraphics[width=\columnwidth]{./fig/eos_comparison}
  \caption{\label{fig:eos} Comparisons of the entropy density $s$, energy density $e$ and pressure $p$ divided by powers of the temperature $T$ for the 
	   three equations of state used in this study.}
\end{figure}

We generate hydrodyamic initial conditions using a standard two-component Monte Carlo Glauber model which deposits entropy proportional to a linear 
combination of nucleon participants and binary nucleon-nucleon collisions,
\begin{equation}
 dS/dy \,\vert_{y=0} \propto \frac{(1-\alpha)}{2}N_\text{part} + \alpha N_\text{coll}.
 \label{twocomponent}
\end{equation}

The entropy is localized about each nucleon's transverse parton density $T_p({\bf x})$,
\begin{eqnarray}
 dS/dy \,\vert_{y=0} &\propto& \sum\limits_{i=0}^{N_\text{part,A}} w_i\, T_p({\bf x} - {\bf x}_i)(1-\alpha + \alpha\, N_\text{coll,i}) \nonumber \\
                     &+& \sum\limits_{j=0}^{N_\text{part,B}} w_j\, T_p({\bf x} - {\bf x}_i)(1-\alpha + \alpha\, N_\text{coll,j}),
 \label{glauber}
\end{eqnarray}
where the summations run over the participants in each nucleus, $N_\text{coll,i}$ denotes the number of binary collisions suffered by the $i^\text{th}$ 
nucleon and the proton density $T_p({\bf x})$ is described by a Gaussian
\begin{equation}
 T_p({\bf x}) = \frac{1}{\sqrt{2 \pi B}} \exp \left(-\frac{x^2+y^2}{2 B} \right)
\end{equation}
with transverse area $B = 0.36$ $\text{fm}^2$.

The random nucleon weights $w_i$ in equation \eqref{glauber} are sampled independently from a Gamma distribution with unit mean
\begin{equation}
 P_k(w) = \frac{k^k}{\Gamma(k)} w^{k-1} e^{-k w},
\end{equation}
and shape parameter $k = \text{Var}(P)^{-1}$ which modulates the variance of the distribution. These fluctuations are typically added \cite{?} to reproduce 
the large multiplicity fluctuations observed in minimum bias proton-proton collisions. In this work the shape parameter is fixed to $k=1$ to fit the $200$ 
GeV UA5 data \cite{?}. 

For the binary collision fraction in equation \eqref{twocomponent}, we choose the value $\alpha=0.14$ used in reference \cite{?}.

It is important to note that the aforementioned Monte Carlo Glauber model is not a state of the art model for initializing hydrodynamic simulations. It 
makes many simplifying assumptions such as ignoring pre-equillibrium dynamicsand asserting wounded nucleon and binary collision scaling, an assertion 
which is questioned by a number of recent works. Nevertheless, the model provides a good description of observed particle multiplicities, flows and spectra. 
In this work, we are primarily interested in the \emph{sensitivity} of hydrodynamic observables to differences in the QGP equation of state and \emph{not} 
the overall best fit of model to data. Hence, the Monte Carlo Glauber model serves as suitable surrogate for more accurate physical models.

\section{Hydrodyamics and \\ Kinetic Description}

The initial condition profiles, which provide the entropy density $dS/(d^2r_\perp\, d\eta\, \tau_\text{therm})$ at the QGP thermalization time, are then scaled
by an overall normalization factor which is tuned to fit the centrality dependence of charged particle production in $200$ GeV gold-gold collisions.

We follow the work in reference \cite{?} and fix the hydrodynamic specific shear viscosity to a constant value $(\eta/s)_\text{QGP}=0.08$ which has been shown to
provide a reasonable description of measured spectra and flows in $200$ GeV gold-gold collisions. For the purposes of this study, we assume vanishing bulk viscosity 
$\zeta/s=0$, although it would be interesting to account for finite bulk viscosity in future work, as its functional form is sensitive to the value of the QGP trace anomaly 
near the QCD phase transition \cite{?}.

In order to switch from hydrodyamic field equations to microscopic transport, the VISHNU model asserts a sudden switching temperature $T_\text{sw}$ at which the 
hydrodynamic energy-momentum tensor is particlized using the Cooper-Frye freezeout prescription,
\begin{equation}
 E\frac{dN_i}{d^3p} = \int_\sigma f_i(x,p) p^\mu d^3\sigma_\mu
 \label{cooper-frye}
\end{equation}
where $f_i$ is the distribution function of particle species $i$, $p^\mu$ is its four-momentum and $d^3\sigma_\mu$ characterizes an element of the isothermal 
freezeout hypersurface defined by $T_\text{sw}$.
 

\section{Equations of State}

The hybrid model approach used in this study 

Within the hybrid model approach used in this study, the low temperature behaviour of the QCD equation of state is fixed by the hadronic transport model.

We concentrate on the intermediate and high temperature dependence of the QCD equation of state and fix the.


QCD equation of state is described by the hadron 
resonance gas model, which is to good approximation



focus on the intermediate and high temperature dependence of the QGP equation of state


study three parameterizations for the QGP equations of state, the state of the art HotQCD and Wuppertal-Budapest parameterizations \cite{?} as well as the 
older s95-v1 parametrization based on older HotQCD results. To perform


\section{Results}

\begin{figure*}[t]
  \includegraphics[width=\textwidth]{./fig/spectra}
  \caption{
    \label{fig:spectra}
  }
\end{figure*}

\begin{figure*}[t]
  \includegraphics[width=\textwidth]{./fig/v2}
  \caption{
    \label{fig:spectra}
  }
\end{figure*}

\begin{figure*}[t]
  \includegraphics[width=\textwidth]{./fig/v3}
  \caption{
    \label{fig:spectra}
  }
\end{figure*}

\begin{figure*}[t]
  \includegraphics[width=\textwidth]{./fig/hbt}
  \caption{
    \label{fig:spectra}
  }
\end{figure*}


\section{Acknowledgements}

\medskip
JSM acknowledges support by the DOE/NNSA Stockpile Stewardship Graduate Fellowship under grant no.~DE-FC52-08NA28752.

\bibliography{eos,duke-qcd-refs/Duke_QCD_refs}


\end{document}
